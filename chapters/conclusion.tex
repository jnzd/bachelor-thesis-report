\chapter{Conclusion}

To recap, we have set out to build a wrapper around MonPoly that handles persistent data storage, offers a web friendly interface to MonPoly, and provides a policy change method.
MonPoly is a runtime monitoring tool that uses MFOTL (and in its newest version also MFODL) as a policy specification language.
MFOTL and MFODL are powerful, as in they can express a large amount of properties.

The amount of computer systems in the world is only growing and with it the need for security guarantees is also increasing.
While MonPoly does not directly offer such guarantees, it is a useful tool in detecting policy violations.

Considering that a runtime monitor and a database storing system activities often rely on the same data, the idea to combine storing events and monitoring them is a natural one.
Some regulation, like GDPR, also requires logging of certain information.
This combination of technologies also offers new possibilities, that were not easily achievable before.
One such thing is a policy change.
When past data can simply be read from the database, it becomes easy to restore the same state when starting to monitor a new policy.

There are many points of future interest around the wrapper specifically and more generally about policy change.
As far as the wrapper is concerned, we have already briefly mentioned that one natural expansion would be to monitor multiple policies at the same time.
The wrapper would be well suited to abstract away the intricacies and provide a simple interface to the user.

An obvious area for future work is our partial policy change, the core idea of which is outlined at the end of chapter 3.
Such an approach may be sufficient to achieve ideal performance.
Data protection, where a user can add or remove certain permissions, can be one such use case.
