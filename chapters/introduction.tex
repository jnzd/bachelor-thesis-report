\section{Introduction}

MonPoly \cite{Basin2017} is a tool for runtime monitoring, also called Runtime Verification (RV) \cite{Bartocci2018}.
Runtime monitoring is a method for checking a systems behaviour against a formal specification during or after execution of the system.
If the monitoring happens during execution this is called online monitoring and if it is done after the system is done executing we call this offline monitoring.
MonPoly is capable of both forms of monitoring.
The monitoring is done on a trace of the system.
A trace refers to a timestamped log of events.

% TODO write possible use case for monitoring and MonPoly

% TODO give overview of the goals of the project
We extended MonPoly with a backend written in Python. The backend connects MonPoly to a time series database (QuestDB).
It facilitates moving the monitoring from one machine to another and efficiently reload the previous state as long as the database is retained. 
We do this by making use of relative intervals and only loading events that are within the relative interval of a formula.
Further the backend enables a first method for changing the monitored policy.




\subsection{Metric First-Order Temporal Logic}
MonPoly uses Metric First-Order Temporal Logic (MFOTL) \cite{Basin2015, Chomicki1995} as a policy specification language. % \cite{Basin2015,Hublet2022}.
It can monitor a subset of possible MFOTL formulas.
MFOTL is a 

\subsection{MonPoly}

\subsection{Time Series Databases}

