\begin{itemize}
    \item \texttt{/}
    This endpoint is designed to be viewed in the browser, and it displays basic information about the state of the wrapper.
    It shows the current policy and signature, if they have been set.
    Perhaps most importantly it displays the output from MonPoly.

    \item \texttt{/set-policy}
    This takes any policy or formula file for MonPoly.
    With this command the policy is only stored by the wrapper.
    Any checks are only done when attempting to start the monitor itself.
    \begin{verbatim}
$ curl -X POST -F policy=@<path/to/policy>
                <wrapper-hostname>[:port]/set-policy
> {"message":"policy set to  <contents of policy file>"}
    \end{verbatim}

    \item \texttt{/change-policy}

    This command can only be issued when the monitor is already running.
    It will only succeed if the new policy provided is monitorable with the signature that was initially provided.
    There is no way to update the signature once the monitoring has begun.
    \begin{verbatim}
       $ curl <wrapper-hostname>/change-policy -f <path/to/policy>
    \end{verbatim}
    \item \texttt{/get-policy}
    This returns the policy as a JSON object, as seen below.
    \begin{verbatim}
$ curl -X GET <wrapper-hostname>[:port]/get-policy

{"policy":"<contents of policy file>"}
    \end{verbatim}

    \item \texttt{/set-signature}
    \begin{verbatim}
$ curl -X POST -F signature=@<path/to/signature> \
                <wrapper-hostname>[:port]/set-signature

{"message":"signature set to <contents of signature file>"}
    \end{verbatim}

    \item \texttt{/get-signature}
    \begin{verbatim}
$ curl -X GET <wrapper-hostname>[:port]/get-signature

{"signature":"<contents of signature file>"}
    \end{verbatim}
    \item \texttt{/start-monitor}
    \begin{verbatim}
$ curl -X GET localhost:5000/start-monitor      
{
    "pid": <process id of MonPoly>,
    "args": [ <command line arguments passed to MonPoly> ]
}
    \end{verbatim}
    \item \texttt{/stop-monitor}
    \item \texttt{/reset-everything}
    \item \texttt{/get-events}
    \item \texttt{/get-most-recent}

\end{itemize}