\chapter{Background}

\section{Metric First-Order Temporal and Dynamic Logic}
As mentioned in the introduction, Metric First-Order Temporal Logic (MFOTL) \cite{Basin2008, Basin2015, Chomicki1995} is used as a policy specification language by MonPoly.
Here we give a quick overview of MFOTL.
MFOTL is well suited to express a variety of policies one might want to monitor.
It combines First Order Logic (FOL) with metric temporal operators.
The metric aspect of these operators is the interval $I$ they are bound by.
This interval denotes a time frame in which the formula needs to be satisfied.


Metric First-Order \textit{Dynamic} Logic (MFODL) \cite{Basin2020} is an even more expressive specification language than MFOTL.
MFODL introduces the notion of regular expressions.
It is more general than MFOTL and in theory the temporal operators from MFOTL could be replaced by the two more powerful operators introduced by MFODL.
In practice, it is often useful to keep the basic temporal operators as we can apply specialized optimizations to them that cannot be done with regular expressions.

Basin et al. \cite{Basin2015aggregations} extends MFOTL with aggregations.
Aggregation operations like SUM are commonly seen in database contexts.
When considering an example like a monthly spending limit for a credit card it becomes clear how aggregations can be useful in policy monitoring.

\subsection{Syntax and Semantics}
Let's recall the syntax and semantics of MFOTL \cite{Basin2008,Basin2015,Schneider2019} as well as MFODL \cite{Basin2020}.
Like Basin et al. \cite{Basin2015} we let $\mathbb{I}$ be the set of nonempty intervals over $\mathbb{N}$.
We use the common interval definition with round brackets for open intervals and square brackets for closed ones, e.g. $[a,b) := \{x \in \mathbb{N} \mid a \leq x < b\}$.
A signature $S$, as defined by Basin et al. \cite{Basin2015}, is a tuple $(C,R,\iota)$.
$C$ is a finite set of constant symbols, $R$ is a finite set of predicates, and $\iota : R \to C$ is a function that assigns each predicate $r \in R$ an arity $\iota(r)$.
$C$ and $R$ are disjoint, i.e. $C \cap R = \emptyset$.
Let $S = (C,R,\iota)$ be a signature and $V$ a finite set of variables with $V \cap (C \cup R) = \emptyset$.
Now we have enough information to restate the definition of an MFOTL formula (Definition 2.1 in Basin et al. \cite{Basin2015}).
Let $\circ$ be any ordering relation over $V \cup C$, i.e. $\approx$, $\prec$, $\preceq$.
Basin et al. \cite{Basin2015} only included the equality relation, the other comparisons were added in \cite{Basin2020}.

\renewcommand{\labelenumi}{(\roman{enumi})}
\begin{definition}
    \label{def:mfotl-formula}
    The MFOTL formulas over $S$ are inductively defined in the following way.
    \begin{enumerate}
        \item For $t,t' \in V \cup C$, $t \circ t'$ is a formula.
        \item For $r \in R$ and $t_1, t_2, \dots, t_{\iota(r)} \in V \cup C$, $r(t_1,t_2,\dots,t_{\iota(r)})$ is a formula.
        \item For $x \in V$, if $\phi$ and $\psi$ are formulas then $(\neg \phi),(\phi \lor \psi),\text{ and }(\exists x . \phi)$ are formulas.
        \item For $I \in \mathbb{I}$, if $\phi$ and $\psi$ are formulas then $(\Previous{I} \phi), (\Next{I} \phi), (\phi \Since{I} \psi), \text{ and }(\phi \Until{I} \psi)$ are formulas.
    \end{enumerate}
\end{definition}

Definition \ref{def:mfotl-formula} omits common first order logical operators $\land$ ("and") and $\forall$ ("for all").
These can be constructed as syntactic sugar from the three FOL operators negation, or, and the exists quantifier.
The metric temporal operators in MFOTL are $\Until{I}$ ("until"), $\Since{I}$ ("since"), $\Previous{I}$ ("previous"), and $\Next{I}$ ("next").
Similarly to FOL the four MFOTL operators can also be used to construct other convenient operators such as $\Once{I}$ ("once"), $\Eventually{I}$ ("eventually"), $\Always{I}$ ("always"), and $\Historically{I}$ ("historically").
See Basin et al. \cite{Basin2015} for the concrete derivations of these additional operators.

The subscript $I$ of the temporal operators specifies the time interval in the past or the future in which a formula must be satisfied.
For the evaluation of a formula Basin et al. \cite{Basin2015} introduces the notion of a structure $\mathcal{D}$ over a signature $S=(C,R,\iota)$.
It consists of the domain $| \mathcal{D} | \neq \emptyset$ and interpretations $c^\mathcal{D} \in |\mathcal{D}|$ and $r^\mathcal{D} \in |\mathcal{D}|$, for each $c \in C$ and $r \in R$.
Basin et al. \cite{Basin2015} further defines a temporal structure as a pair $(\bar{\mathcal{D}}, \bar{\tau})$ whith $\bar{\mathcal{D}} = (\mathcal{D}_0, \mathcal{D}_1, \dots)$ is a sequence of structures over $S$ and $\bar{\tau} = (\tau_0, \tau_1, \dots)$ is a sequence of non-negative integers (\textit{time stamps}) with the following 3 properties.
\renewcommand{\labelenumi}{\arabic{enumi}.}
\begin{enumerate}
    \item The sequence is monotonically increasing, i.e. $\forall i . \tau_i \leq \tau_{i+1}$. Moreover $\bar{\tau}$ makes progress for every $\tau \in \mathbb{N}$ there is some $i$ such that $\tau < \tau_i$. 
    \item $\bar{\mathcal{D}}$ has constant domains, that is, for all $i \geq 0$, $|\mathcal{D}_i| = |\mathcal{D}_{i+1}|$.
    \item Each constant symbol $c \in C$ has a rigid interpretation, that is, for all $i \geq 0$, $c^{\mathcal{D}_i} = c^{\mathcal{D}_{i+1}}$.
\end{enumerate}
A temporal structure is sometimes called a trace and denoted $\sigma$.
It is important to note the difference between \textit{time stamps} and \textit{time points}.
Time points are the combination of a time stamp $\tau_i$ and a structur $\mathcal{D}_i$, where $i$ is the index by which a time point is described.
Since the domain of the structures $\mathcal{D}_i$ and the valuation of constants $c \in C$ do not change they are denoted as $|\bar{\mathcal{D}}|$ and $c^{\bar{\mathcal{D}}}$ respectively.
Basin et al. defines a \textit{valuation} as a mapping $v : V \to |\bar{\mathcal{D}}|$.
And for a variable vector $\bar{x} = (x_1,\dots,x_n)$, and $\bar{d}=(d_1,\dots,d_n) \in |\bar{\mathcal{D}}$ it defines the notation $v[\bar{x} \mapsto \bar{d}]$ for the valuation that maps $x_i$ to $d_i$ for $1 \leq i \leq n$.
Further it uses a notational hack to also apply a valuation $v$ to constant symbols $c \in C$ where $v(c) = c^{\bar{\mathcal{D}}}$.
For convenience we restate Definition 2.2 from Basin et al. \cite{Basin2015}.
Since we changed the definition of a formula to include ordering relations ($\prec, \preceq$) instead of only equality ($\approx$), we append this definition to allow for that.
These additions are also in Figure 1 of Basin et. al \cite{Basin2020}.

\begin{definition}
    \label{def:mfotl-evaluation}
    Let $(\bar{\mathcal{D}}, \bar{\tau})$ be a temporal structure over the signature $S = (C,R,\iota)$, with $\bar{\mathcal{D}} = (\mathcal{D}_0,\mathcal{D}_1,\dots)$ and $\bar{\tau} = (\tau_0, \tau_1, \dots)$, $\phi$ a formula over $S$, $v$ a valuation, and $i \in \mathbb{N}$.
    We define the relation $(\bar{\mathcal{D}},\bar{\tau},v,i) \models \phi$ inductively as follows.
    \begin{alignat*}{3}
        &(\bar{\mathcal{D}},\bar{\tau},v,i) \models t \approx t'  
            &&\text{iff } && v(t) = v(t') \\
        &(\bar{\mathcal{D}},\bar{\tau},v,i) \models t \prec t'  
            &&\text{iff } && v(t) < v(t') \\
        &(\bar{\mathcal{D}},\bar{\tau},v,i) \models t \preceq t'  
            &&\text{iff } && v(t) \leq v(t') \\
        &(\bar{\mathcal{D}},\bar{\tau},v,i) \models r(t_1,\dots,t_{\iota(r)})  
            \qquad
            &&\text{iff } && (v(t_1), \dots, v(t_{\iota(r)})) \in r^{\mathcal{D}_i} \\
        &(\bar{\mathcal{D}},\bar{\tau},v,i) \models (\neg \psi) 
            &&\text{iff } && (\bar{\mathcal{D}},\bar{\tau},v,i) \not\models \psi \\
        &(\bar{\mathcal{D}},\bar{\tau},v,i) \models (\psi \lor \psi') 
            &&\text{if } && (\bar{\mathcal{D}},\bar{\tau},v,i) \models \psi 
            \text{ or } (\bar{\mathcal{D}},\bar{\tau},v,i) \models \psi' \\
        &(\bar{\mathcal{D}},\bar{\tau},v,i) \models (\exists x. \psi) 
            &&\text{iff } && (\bar{\mathcal{D}},\bar{\tau},v[x \mapsto d],i) \models \psi 
            \text{ for some } d \in |\bar{\mathcal{D}}| \\
        &(\bar{\mathcal{D}},\bar{\tau},v,i) \models (\Previous{I} \psi) 
            &&\text{iff } && i > 0, \tau_i - \tau_{i-1} \in I, \text{ and } (\bar{\mathcal{D}}, \bar{\tau}, v, i-1) \models \psi \\
        &(\bar{\mathcal{D}},\bar{\tau},v,i) \models (\Next{I} \psi)
            &&\text{iff } && \tau_{i+1} - \tau_i \in I \text{ and } (\bar{\mathcal{D}},\bar{\tau},v,i+1) \models \psi \\
        &(\bar{\mathcal{D}},\bar{\tau},v,i) \models (\psi \Since{I} \psi')
            &&\text{iff} && \text{for some } j \leq i, \tau_i - \tau_j \in I, (\bar{\mathcal{D}}, \bar{\tau}, v, j) \models \psi' \text{ and } \\
            &&&&&(\bar{\mathcal{D}}, \bar{\tau}, v, k) \models \psi \text{ for all } k \in \mathbb{N} \text{ with } j < k \leq i \\
        &(\bar{\mathcal{D}},\bar{\tau},v,i) \models (\psi \Until{I} \psi')
            &&\text{iff } && \text{for some } j \geq i, \tau_j - \tau_i \in I, (\bar{\mathcal{D}}, \bar{\tau}, v, j) \models \psi' \text{ and } \\
            &&&&&(\bar{\mathcal{D}}, \bar{\tau}, v, k) \models \psi \text{ for all } k \in \mathbb{N} \text{ with } i \leq k < j \\
    \end{alignat*}
\end{definition}

Anologous to Definition \ref{def:mfotl-evaluation} we now give the definition of an MFODL formula.
The definition is based on Figure 4 of Basin et al. \cite{Basin2020}, but in a notation closer to the one we already used for MFOTL formulas.
For MFODL formulas we first have to define regular expressions, as MFODL formulas depend on them.
\renewcommand{\labelenumi}{(\roman{enumi})}
\begin{definition}
    \label{def:mfodl-formula}
    The MFODL regular expressions are defined in the following way:
    \begin{enumerate}
        \item If $k \in \mathbb{N}$ then $\star^k$ is a regular expression.
        \item If $\phi$ is a formula then $(\phi ?)$ is a regular expression.
        \item If $\rho$ is a regular expression then $(\rho^*)$ is a regular expression.
        \item If $\rho$ and $\sigma$ are regular expressions then $(\rho + \sigma)$ and $(\rho \cdot \sigma)$ are regular expressions.
    \end{enumerate}
\end{definition}
With this we extend Definition \ref{def:mfotl-formula} to include MFODL formulas.
\begin{definition}
    The MFODL formulas over $S$ are inductively defined in the following way.
    \label{def:regex}
    \begin{enumerate}
        \item MFOTL formulas are also MFOTL formulas
        \item For $I \in \mathbb{I}$, if $\rho$ is a regular expression then $(\Pregex{I} \rho)$ and $(\Fregex{I} \rho)$ are formulas.
    \end{enumerate}
\end{definition}
For the evaluation of an MFODL formula we first need to define how regular expressions get evaluated.
The definition is originally given through the \texttt{match} function in Figure 4 of Basin et al. \cite{Basin2020}.
We formulate the definition that is more in line with the mathematical notation we used this far.
\begin{definition}
    \label{def:regax-eval}
    Let $(\bar{\mathcal{D}}, \bar{\tau})$ be a temporal structure over the signature $S = (C,R,\iota)$, with $\bar{\mathcal{D}} = (\mathcal{D}_0,\mathcal{D}_1,\dots)$ and $\bar{\tau} = (\tau_0, \tau_1, \dots)$, $\phi$ a formula over $S$, $v$ a valuation, and $i,j \in \mathbb{N}$.
    \begin{alignat*}{3}
        &(\bar{\mathcal{D}}, \bar{\tau}, v, i, j) \modelsreg \star^k
            &&\text{iff } && i = j + k\\
        &(\bar{\mathcal{D}}, \bar{\tau}, v, i, j) \modelsreg \phi ?
            &&\text{iff } && (\bar{\mathcal{D}}, \bar{\tau}, v, i) \models \phi \text{ and } i = j\\
        &(\bar{\mathcal{D}}, \bar{\tau}, v, i, j) \modelsreg \rho + \sigma \qquad
            &&\text{iff } && (\bar{\mathcal{D}}, \bar{\tau}, v, i, j) \modelsreg \rho \text{ or }(\bar{\mathcal{D}}, \bar{\tau}, v, i, j) \modelsreg \sigma \\
        &(\bar{\mathcal{D}}, \bar{\tau}, v, i, j) \modelsreg \rho \cdot \sigma \qquad
            &&\text{iff } && \text{ for some } k \in \mathbb{N}, (\bar{\mathcal{D}}, \bar{\tau}, v, i, k) \modelsreg \rho \text{ and }(\bar{\mathcal{D}}, \bar{\tau}, v, k, j) \modelsreg \sigma \\
        &(\bar{\mathcal{D}}, \bar{\tau}, v, i, j) \modelsreg \rho^*
            &&\text{iff } && \text{ for some } l \in \mathbb{N}, o_1,\dots,o_l \in \mathbb{N}, (\bar{\mathcal{D}}, \bar{\tau}, v, i, o_1) \modelsreg \rho \text{ and } \dots  \\
            &&&&&\text{ and } (\bar{\mathcal{D}}, \bar{\tau}, v, o_l, j) \modelsreg \rho \\
    \end{alignat*}
    
\end{definition}
With the help of Definition \ref{def:regax-eval} we can now define the evaluation of MFODL formulas which is also given in Figure 4 of Basin et al. \cite{Basin2020} through the \texttt{sat} function.
The trace $\sigma$ in the \texttt{sat} function is simply our temporal structure $(\bar{\mathcal{D}},\bar{\tau})$.
The data list $v$ is the valuation mapping.
And $i$ specifies the current time point in both notations.
\begin{definition}
    \label{def:mfodl-eval}
    Let $(\bar{\mathcal{D}}, \bar{\tau})$ be a temporal structure over the signature $S = (C,R,\iota)$, with $\bar{\mathcal{D}} = (\mathcal{D}_0,\mathcal{D}_1,\dots)$ and $\bar{\tau} = (\tau_0, \tau_1, \dots)$, $\phi$ a formula over $S$, $v$ a valuation, and $i \in \mathbb{N}$.
    We define the relation $(\bar{\mathcal{D}},\bar{\tau},v,i) \models \phi$ inductively as follows.
    \begin{alignat*}{3}
        &(\bar{\mathcal{D}},\bar{\tau},v,i) \models \Pregex{I} \rho
            &&\text{ iff } && \text{for some } j \leq i, \tau_i - \tau_j \in I \text{ and }(\bar{\mathcal{D}}, \bar{\tau}, v, j, i) \modelsreg \rho \\
        &(\bar{\mathcal{D}},\bar{\tau},v,i) \models \Fregex{I} \rho
            &&\text{ iff } && \text{for some } j \geq i, \tau_j - \tau_i \in I \text{ and }(\bar{\mathcal{D}}, \bar{\tau}, v, i, j) \modelsreg \rho \\
    \end{alignat*}
    
\end{definition}


\section{MonPoly}
MonPoly \cite{Basin2017} is a policy monitoring tool written in OCaml that supports MFOTL with aggregations and in its newest iterations it also has support for MFODL.
MonPoly can monitor a fragment of MFOTL/MFODL where all future operators must be bounded.
One major exception to that rule is an (implicit) always operator around the desired policy.

Let's return to the social media example from the introduction and look at how we would go about monitoring that policy with MonPoly.
We recall our description in words:
    "If a user's location data is accessed and the purpose of the access is for tailoring advertisements, the user must have previously given permission for their location data to be used for advertising purposes."
In MonPoly a policy is tied to a signature.
A signature can be compared with a database schema and describes the arity and types of possible events.
So let's consider a possible signature for our example: 

\begin{figure}[H]
    \label{fig:example-signature}
\begin{verbatim}
loc_accessed(user_id: int, purpose: string)
perm_granted(user_id: int)
perm_revoked(user_id: int)
\end{verbatim}
    \caption{Example MonPoly Signature}
\end{figure}

This is a basic signature with 3 predicates.
The first one means that a users location data has been used for a specified purpose.
The last two events get triggered when a user either grants or revokes permission for their location data to be used for advertising purposes.
Let's now define the policy in a formal manner.
\begin{align*}
    \Always (\texttt{loc\_accessed(i, "advertising")}
    \implies (\Once{[0,\infty)} &\texttt{perm\_granted(i)} \\
         \land  \neg (\texttt{perm\_revoked(i)} \Since{[0,\infty)} 
            &\texttt{perm\_granted(i)})))
\end{align*}

For MonPoly we first get rid of the surrounding $\Always{}$, because MonPoly implicitly adds an always-operator around any policy.
The remaining formula in MonPoly syntax is the following:

\begin{figure}[H]
    \label{fig:example-policy}
\begin{verbatim}
loc_accessed(i, "advertising")
IMPLIES 
(
    (ONCE[0,*) perm_granted(i)) 
    AND 
    (NOT (perm_revoked(i) SINCE[0,*) perm_granted(i)))
)
\end{verbatim}
    \caption{Example MonPoly Policy}
\end{figure}

While MonPoly cannot actually monitor this formula directly, it can monitor the negation of this formula.
For this on can use the \texttt{-negate} flag when running MonPoly.

\section{Time Series Databases}

Time series databases are a class of databases optimized for timestamped data.
For example, they optimize for data retrieval within a certain time range.
With the advance of internet of things devices with built-in sensors time series databases are experiencing explosive growth.
And as we have established they happen to fit well with our monitoring goals.
There are many different options of time series databases available.
We were looking for something with good performance, good support for tables of data, and good usability.
We have opted for QuestDB \cite{questdb}.

\subsection{QuestDB}
QuestDB uses a column-based storage model \cite{questdb-storage-model}.
It supports the PostgreSQL wire protocol \cite{questdb-postgres-wire} for querying and inserting data.
This is the same protocol used by the popular relational database PostgreSQL \cite{postgres}.
The support for SQL as a query language is of great use, because SQL is the most widely used and taught database query language.
PostgreSQL is in itself a very popular dialect of SQL and thus has a lot of tooling available which can also be used with QuestDB.
It further provides a REST API \cite{questdb-rest} and has a web console for both inserting and querying data.
For best performance it supports the InfluxDB Line Protocol \cite{questdb-influx-db-line-protocol, influx-line} with client libraries for most popular modern programming languages.
InfluxDB is itself a time series database and the line protocol is a write-protocol developed by and for InfluxDB optimized for time series data.
The QuestDB client libraries for different programming languages are implementations of the InfluxDB line protocol by QuestDB.
QuestDB recommends using the Line Protocol for the best write performance, compared to the other methods of writing data.
The Line Protocol cannot be used to query data.
For that the PostgrSQL Wire Protocol is the recommended method.
QuestDB is written in Java, open source, and licensed under the Apache 2.0 license.

\subsection{Time Series Databases for Runtime Monitoring}
In this section we explore how a timeseries database can be utilized in the context of runtime monitoring.
We will contrast policies and database queries, as well as signatures and database schemas.
An important and possibly counter intuitive thing is that MFOTL is not any more expressive than FOL.
In fact they have the same expression strength, i.e. any MFOTL formula can be expressed as a FOL formula and vice versa.
Any FOL formula is by definition also a MFOTL formula.
The other way around any trace in MFOTL can be simulated in FOL by extending predicates with two attributes for the time point and time stamp.
Take for example the formula $\phi = \Previous{[a,b)} P()$ where $P$ is a predicate with arity $0$.
$\phi$ is equivalent to the FOL formula $\phi' = P(t_s, t_p) \land \exists t_s' . P(t_s', t_p - 1) \land a \leq t_s - t_s' < b$.

It is relatively straight forward to create a database schema from a signature like the one in Figure \ref{fig:example-signature}.
We create one table per predicate.
The predicate name is used as the table name.
The columns and their types are given by the attributes.
MonPoly allows for optional attribute names, we disregard those, if they are present and rely on the ordering of attributes.
We need two additional columns, one for the time point, i.e. an integer index, and one for the time stamp when the event occurred.
One additional table keeping track of all time points and their time stamps is necessary, because a time point could occur without any event attached to it, if we only had the events stored, we would lose any record of such time points.
And any time point can influence the evaluation of a formula.
This time stamp and time point table technically makes it redundant to store the time stamps with every predicate, since the mapping from time point indices to time stamps is injective.
But QuestDB requires a time stamp column, and it is also beneficial for performance as QuestDB is optimized for queries on time ranges.

The database schema as SQL create statements for our example policy from Figure \ref{fig:example-policy} can be seen in Figure \ref{fig:example-policy-schema}.
The \texttt{timestamp()} function is a special extension to SQL for QuestDB.
It specifies the column that is used as the dedicated time stamp in a table.
If it is not specified an additional time stamp column would get added.
% The \texttt{PARTITION BY DAY} part tells QuestDB how to partition the tables.
By default, no partitioning would be used.

\begin{figure}[H]
    \label{fig:example-policy-schema}
\begin{verbatim}
CREATE TABLE perm_revoked(x1 INT,
                          time_stamp TIMESTAMP,
                          time_point INT) 
                          timestamp(time_stamp);
CREATE TABLE perm_granted(x1 INT,
                          time_stamp TIMESTAMP,
                          time_point INT) 
                          timestamp(time_stamp);
CREATE TABLE loc_accessed(x1 INT, x2 STRING,
                          time_stamp TIMESTAMP,
                          time_point INT) 
                          timestamp(time_stamp);
CREATE TABLE ts( time_stamp TIMESTAMP,
                 time_point INT) 
                 timestamp(time_stamp);
\end{verbatim}
    \caption{SQL Schema for Example Policy}
\end{figure}


 


