\section{Background}

\subsection{Metric First-Order Temporal Logic}
As mentioned in the introduction, Metric First-Order Temporal Logic (MFOTL) \cite{Basin2008, Basin2015, Chomicki1995} is used as a policy specification language by MonPoly.
Here we give a quick overview of MFOTL.
MFOTL well suited to express a variety of policies one might want to monitor.
It combines First Order Logic (FOL) with metric temporal operators.
FOL provides us with common logic operators like $\vee$ ("and"), $\lor$ ("or"), and $\neg$ ("not") as well as quantifiers $\forall$ ("for all") and $\exists$ ("exists").
The metric temporal operators in MFOTL are $\Until_I$ ("until"), $\Since_I$ ("since"), $\Previous_I$ ("previous"), and $\Next_I$ ("next").
These operators can be used to construct further syntactic sugar operators such as $\Once_I$ ("once"), $\Eventually_I$ ("eventually"), $\AlwaysF_I$ ("always from now on"), and $\AlwaysP_I$ ("always until now").
The metric aspect of these operators is the interval $I$ they are bound by.
This interval denotes a time frame in which the formula needs to be satisfied.

Metric First-Order \textit{Dynamic} Logic (MFODL) \cite{Basin2020} is an even more expressive specification language than MFOTL.
MFODL introduces the notion of regular expressions.
For an exact definition of these regular expressions and the two new operators they introduce see figure 4 of Basin et al. 2020 \cite{Basin2020}.
Similarly to how $\Once_I$, $\Eventually_I$, $\AlwaysF$, and $\AlwaysP$ can be derived from the four core operators $\Until_I$, $\Since_I$, $\Previous_I$, and $\Next_I$, these core operators could theoretically be replaced by the two new regular expression operators.
The exact conversion can also be seen in Basin et al. 2020 \cite{Basin2020}.
In practice, it is often useful to keep the basic temporal operators as we can apply specialized optimizations to them that cannot be done with regular expressions.

In Basin et al. 2015 \cite{Basin2015aggregations} they extended MFOTL with aggregations.
Aggregation operations like SUM are commonly seen in database contexts.
When considering an example like a monthly spending limit for a credit card it becomes clear how aggregations can be useful in policy monitoring.


\subsection{MonPoly}


\subsection{Time Series Databases}