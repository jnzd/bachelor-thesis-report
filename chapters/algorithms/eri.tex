\section{Relative Interval Extension}

This idea of relative intervals can already filter an existing trace down to a much smaller one by removing events that are unnecessary for the evaluation of a given policy.
We expand on this by creating and using a data structure that allows us to select an even smaller sub trace with the same effect of not changing the truth value of the policy.

First we move from one relative interval for an entire policy to one relative interval per predicate occurring in a policy.
We break this down further.
Every predicate comes with a number of attributes as defined in the signature.
Some attributes are potentially constant.

Looking back at our policy from Figure \ref{fig:example-policy}, \texttt{"advertising"} is one such constant attribute in the predicate \texttt{loc\_accessed}.
This means any occurrence of the predicate \texttt{loc\_accessed} where the second attribute is not \texttt{"advertising"}, has no influence on our policy and is therefore not needed in a potential sub trace.
We check every predicate in our policy for constant attributes.
Then we take the set of different arrangements of constant and variable attributes per predicate.
We define a structure that captures constant and variable attributes of a predicate.
\begin{definition}
    \label{def:mask}
    Let $S=(C,R,\iota)$ be a signature and $r \in R$ a predicate with arity $\iota(r)$.
    A mask for the predicate $r$ is a tuple $m=(m_1, \dots, m_{\iota(r)})$ with $m_1, \dots, m_{\iota(r)} \in C \cup \{v\}$ and $v \not\in V \cup C$.
\end{definition}
$v$ is a placeholder value denoting attributes in the mask that have a non-constant value.
Each mask has its own relative interval.
For our example the masks with their corresponding relative intervals can be seen in Figure \ref{fig:example-ext-intervals}.

\begin{figure}[H]
    \label{fig:example-ext-intervals}
\begin{verbatim}
loc_accessed(*,"advertising") -> [0,0]
perm_granted(*) -> (*,0]
perm_revoked(*) -> (*,0]
\end{verbatim}
    \caption{Extended Relative Intervals of Example Policy}
\end{figure}

A \texttt{*} (asterisk) in the attributes denotes a variable value, $v$ in Definition \ref{def:mask}.
In larger formulas there can be multiple different masks per predicate.
Let $\mathcal{M}(r)$ be the set of possible masks for a predicate $r$.

\begin{definition}
    \label{def:mask-precision-comparison}
    Let $m = (m_1, \dots, m_{\iota(r)})$ and $n = (n_1, \dots, n_{\iota(r)})$ be two masks for a predicate $r \in R$, where $R$ is part of the signature $S = (C,R,\iota)$.
    $m \nomoreprecise n$ denotes that $m$ is no more precise than $n$, i.e. for all $i \in \mathbb{N}$ with $0 \leq i \leq \iota(r)$, $m_i = n_i$ or $m_i = v$.
\end{definition}

We use a map data structure to store the predicates, masks, and their respective relative intervals.

\begin{definition}
    \label{def:map}
    Let $S=(C,R,\iota)$ be a signature and let $\mathcal{M} : R \to \{(C \cup \{v\})^*\}$ be the function $\mathcal{M}(r) = M$ that gives the set of all possible masks $M$ for a predicate $r$.
    An \textbf{masked predicate map} is a set $\{(k,i)\}$ where $k = (r,m)$ with $r \in R$ and $m \in \mathcal{M}(r)$ and $i \subseteq \mathbb{Z}$ is an interval over $\mathbb{Z}$. 
\end{definition}


On this data structure we define the operators $\Cupmerge$, $\Cupext$ and $\oplusext$.

\begin{definition}
    \label{def:e-rel-int-ops}
    Let $M$ and $N$ be two masked predicate maps and $T$ a positive interval, then 
    \begin{align*}
        M \Cupmerge N = 
            & \{ p(l) \rightarrow (I \Cup J) \mid 
                p(l) \rightarrow I \in m \text{ and } 
                p(l) \rightarrow J \in n \} \\
            & \cup \{p(l) \rightarrow I \mid  
                (p(l) \rightarrow I \in m \text{ and }
                p(l) \in \keys(M) \setminus \keys(N)) \} \\
            & \cup \{p(l) \rightarrow I \mid  
                (p(l) \rightarrow I \in n \text{ and }
                p(l) \in \keys(N) \setminus \keys(M))
                \}        
                \\
        T \Cupext M = 
            & \{ p(l) \rightarrow (T \Cup I) \mid 
                p(l) \rightarrow I \in M \} \\
        T \oplusext M = 
            & \{ p(l) \rightarrow (T \oplus I) \mid 
                p(l) \rightarrow I \in M \} \\
    \end{align*}
\end{definition}

The notation $p(l) \rightarrow I$ denotes an element in a masked predicate map.
It is equivalent to the notation $((p,l), I)$, but it better shows how we are working with a map.
The $\keys$ operator gives all the first elements, the predicate name and mask tuples, in a masked predicate map.
With the help of the operators $\Cupmerge$, $\Cupext$ and $\oplusext$ we now give a recursive definition for our extension of relative intervals.

We define a relation on masked predicate maps that is analogous to the subset-equals relation for intervals and illustrates that one map is contained in another.

\begin{definition}
    \label{def:subseteqmap}
    Let $M$ and $N$ be two masked predicate maps.
    \begin{equation*}
        N \subseteqmap M
    \end{equation*}
    if for all $((p,l) \to I) \in N$ there exists $((p',l') \to I') \in M$ with
    \begin{equation*}
        (p = p') \land (l' \nomoreprecise l) \land (I \subseteq I')
    \end{equation*}
\end{definition}

\begin{lemma}
    \label{lem:map-union-subsets}
    Let $M$ and $N$ be masked predicate maps, then
    \begin{equation*}
        M \subseteqmap M \Cupmerge N
    \end{equation*}
    and
    \begin{equation*}
        N \subseteqmap M \Cupmerge N.
    \end{equation*}
\end{lemma}
\textit{Proof}
Since the definition of $M \Cupmerge N$ is symmetric, it suffices to show on of the two relations in the lemma.
From Definition \ref{def:e-rel-int-ops} it follows that any key consisting of a predicate and a mask is also in $M \Cupmerge N$.
Since the mask is there unaltered, it satisfies the condition of being no more precise than the one in $M$.
The condition for $(I \subseteq I')$ is apparent from the definition as well, no interval is discarded and all intervals are either enlarged by the special union of intervals or kept the same.


\begin{definition}
    \label{def:e-rel-int}
    The extended relative interval of the formula $\varphi$, $\ERI(\varphi)$ is defined recursively over the formula structure:
    \begin{equation*}
        \ERI(\varphi) = \\
        \begin{cases}
            \{\} 
                & \text{if $\varphi$ is an an atomic formula} \\ &\text{and not a predicate,} \\ 
            \{p(m) \rightarrow [0,0]\} 
                & \text{if $\varphi$ is a predicate with name } \\ &\text{$p$ and mask $m$,} \\
            \ERI(\psi) 
                & \text{if $\varphi$ is of the form $\neg \psi, \exists x.\psi$,} \\
                & \text{or $\forall x.\psi$,} \\
            \ERI(\psi) \Cupmerge \ERI(\chi) 
                & \text{if $\varphi$ is of the form $\psi \lor \chi$,} \\ & \text{or $\psi \land \chi$,} \\
            (-b,0] \Cupext ((-b,-a] \oplusext \ERI(\psi)) 
                & \text{if $\varphi$ is of the form $\Previous{[a,b)}\psi$,} \\
            [0,b) \Cupext ([a,b) \oplusext \ERI(\psi)) 
                & \text{if $\varphi$ is of the form $\Next{[a,b)}$,}\\
            (-b,0] \Cupext ((-b,0] \oplusext \ERI(\psi)) \Cupmerge ((-b,-a] \oplusext \ERI(\chi)) 
                & \text{if $\varphi$ is of the form $\psi \Since{[a,b)} \chi$,} \\
            [0,b) \Cupext ([0,b) \oplusext \ERI(\psi)) \Cupmerge ([a,b) \oplusext \ERI(\chi)) 
                & \text{if $\varphi$ is of the form $\psi \Until{[a,b)} \chi$,} \\
            [0,b) \Cupext ([0,b) \oplusext \ERIr(\psi)) 
                & \text{if $\varphi$ is of the form $\Fregex{[a,b)} \psi$, and}\\
            (-b,0] \Cupext ((-b,0] \oplusext \ERIr(\psi)) 
                & \text{if $\varphi$ is of the form $\Pregex{[a,b)} \psi$.}\\
        \end{cases}
    \end{equation*}
\end{definition}

And for regular expressions we define 
\begin{definition}
    \label{def:e-rel-int-reg}
    The extended relative interval of the regular expression $\rho$, $\ERIr(\rho)$ is defined recursively over the structure of the regular expression:
    \begin{equation*}
        \ERIr(\rho) =
        \begin{cases}
            \{\} & \text{if $\rho$ is of the form $\star^k$,} \\
            \ERI(\varphi) & \text{if $\rho$ is of the form $\varphi ?$,} \\
            \ERIr(\sigma) \Cupmerge \ERIr(\tau) & \text{if $\rho$ is of the form $\sigma + \tau$ or $\sigma \cdot \tau$, and} \\
            \ERIr(\sigma) & \text{if $\rho$ is of the form $\sigma^*$.}
        \end{cases}
    \end{equation*}
\end{definition}


\subsection{Correctness}
Analogously to regular intervals we want to use the extended relative intervals to extract (slice) a sub trace from a larger trace for a given formula that has the property that the sub trace is sufficient to evaluate the formula.
We first need a few definitions and lemmas before we can proof this property.
\begin{definition}
    \label{def:matching-predicate}
    Let $\mathcal{D}$ be a structure over the signature $S = (C,R,\iota)$, $r^{\mathcal{D}} \in |\mathcal{D}|$ the set of interpretations for the predicate $r \in R$, and $m$ a mask for $r$.
    The arity of $r$, $\iota(r)$ is the same as the length of the mask $m$, $|m|$.
    The mask $m = (m_1, \dots, m_{\iota(r)})$ matches the interpretation $k = (k_1, \dots, k_{\iota(r)}) \in r^{\mathcal{D}}$ if for all $i \in [1,\iota(r)]$ either $m_i = v$ or $m_i = k_i$, where $v$ is again the placeholder value for variable values.
    We write $m \matches k$ for $m$ matches $k$.
\end{definition}

\begin{lemma}
    \label{lem:mask-precision-lemma}
    Let $m = (m_1, \dots, m_{\iota(r)})$ and $n = (n_1, \dots, n_{\iota(r)})$ be two masks for a predicate $r \in R$ with $m \nomoreprecise n$, where $R$ is part of the signature $S = (C,R,\iota)$.
    Then for every interpretation $x$, $n \matches x$ implies that also $m \matches x$.
\end{lemma}
\textit{Proof} 
Note the order of appearance of $m$ and $n$ in the implication is different from that in the relation.
From \ref{def:mask-precision-comparison} it follows that for all $i \in \mathbb{N}$ with $0 \leq i \leq \iota(r)$, either $m_i = n_i$ or $m_i = v$.
If $n \matches k$ then for all $i \in \mathbb{N}$ with $0 \leq i \leq \iota(r)$ either $n_i = v$ or $n_i = k_i$.
We now show that for all $i$ with $i \in \mathbb{N}$ and $0 \leq i \leq \iota(r)$ it also holds that either $m_i = k_i$ or $m_i = v$ by making a case distinction on the two possible values for each $n_i$.
We make a case distinction on these two options
\begin{itemize}
    \item $n_i = v$. In this case $m_i = v$ which satisfies one of the two options of $m_i = k_i$ or $m_i = v$.
    \item $n_i = k_i$. In this case $m=n_i$ or $m=v$ are both possible, but for both options either $m=n_i=k_i$ or $m=v$.
\end{itemize}
We can thus conclude that for all $i \in \mathbb{N}$ with $0 \leq i \leq \iota(r)$, $m_i = k_i$ or $m_i = v$ and thus $m \matches k$.

\begin{definition}
    \label{def:filter}
    Let $M$ be a masked predicate map, $\tau_i, tau_j \in \mathbb{N}$ two time stamps, and $\mathcal{D}$ a structure over the signature $S = (C,R,\iota)$.
    Let $\mathbb{D}$ be the set of all structures over the signature $S$.
    $\filter(M, \mathcal{D}, \tau_i, \tau_j) : \mathbb{D} \to \mathbb{D}$ is a function filtering the structure $\mathcal{D}$ on the condition whether $\tau_i$ is in the relative intervals of the masked predicates in $M$ shifted by $\tau_j$.
    Constant interpretations remain unchanged by the the filtering.
    Concretely for all $r \in R$, 
    \begin{equation*}
        r^{\filter(M, \mathcal{D}, \tau_i, \tau_j)} 
        = \{v(\bar{t}) \in r^{\mathcal{D}} \mid 
            \exists ((p,m), I) \in M. 
                p = r \land m \matches v(\bar{t}) 
                \land \tau_i - 
                \tau_j \in I \}.
    \end{equation*}
\end{definition}
Intuitively $\tau_j$ can be seen as the reference time stamp and we check whether a time stamp $\tau_i$ lies inside the relative intervals from the perspective of $\tau_j$.
The following lemma states that filtering a structure twice has the same effect as filtering once.

\begin{lemma}
    \label{lem:double-filter}
    Let $\mathcal{D}$ be a structure, let $M$ be a masked predicate map and let $\tau_i, \tau_J \in \mathbb{N}$ be two time stamps.
    Then 
    \begin{equation*}
        \filter(M, \filter(M, \mathcal{D}, \tau_i, \tau_j), \tau_i, \tau_j) = \filter(M, \mathcal{D}, \tau_i, \tau_j).
    \end{equation*}
\end{lemma}
\textit{Proof} From Definition \ref{def:filter} we have that the constant interpretations remain unchanged by the filter operation.
It remains to show that the left-hand side (LHS) and the right-hand side (RHS) in Lemma \ref{lem:double-filter} have the same interpretations for predicates.
By Definition \ref{def:filter} we have 
for all $r \in R$,
\begin{equation*}
    r^{\filter(M, \mathcal{D}, \tau_i, \tau_j)} 
    = \{v(\bar{t}) \in r^{\mathcal{D}} \mid 
        \exists ((p,m), I) \in M. 
            p = r 
            \land m \matches v(\bar{t}) 
            \land \tau_i - \tau_j \in I \}.
\end{equation*}
And also for all $r \in R$,
\begin{align*}
    &r^{\filter(M, \filter(M, \mathcal{D}, \tau_i, \tau_j), \tau_i, \tau_j)} \\
    &= \{v(\bar{t}) \in r^{\filter(M, \mathcal{D}, \tau_i, \tau_j)} \mid 
        \exists ((p,m), I) \in M. 
            p = r 
            \land m \matches v(\bar{t}) 
            \land \tau_i - \tau_j \in I \}.
\end{align*}
From this it is apparent that the second set of interpretations, the one filtered twice is a subset of the first one.
Further by them having the same conditional part, if an interpretation $v(\bar{t})$ is in $r^{\mathcal{D}}$ it must also be in $r^{\filter(M, \mathcal{D}, \tau_i, \tau_j)}$, because it must have already satisfied the same condition to be in $r^{\mathcal{D}}$.


Basin et al. \cite{Basin2016} defines a \textit{slice} of a temporal structure in Definition 3.1.
We restate this definition here.
(In the definition $R$ is part of a signature $S=(C,R,\iota)$.)
\begin{definition}
    \label{def:slice}
    Let $s : [0,l) \to \mathbb{N}$ be a strictly increasing function, with $l \in \mathbb{N} \cup \{\infty\}$.
    The temporal structure $(\bar{\mathcal{D}}', \bar{\tau}')$ is a \textit{slice} of $(\bar{\mathcal{D}}, \bar{\tau})$ (with respect to the function $s$) if $\tau_{i'} = \tau_{s(i)}$ and $r^{\mathcal{D}'_i} \subseteq r^{\mathcal{D}_{s(i)}}$, for all $i \in [0,l)$ and all $r \in R$.
\end{definition}

Basin et al. \cite{Basin2016} further defines a $T$-slice in Definition A.2 for slicing a temporal structure in a temporal manner.
We define an analogous notion of an $MT_{\tau}$- slice which is a more fine-grained way of slicing that makes use of masked predicate maps.
\begin{definition}
    \label{def:mt-slice}
    Let $T \subseteq \mathbb{Z}$ be an interval,
        % $\tau \in \mathbb{N}$ a time stamp,
        and $M$ a masked predicate map
            where for all $(m, J) \in M$, $J \subseteq T$.
    And let $(\bar{\mathcal{D}}, \bar{\tau})$ a temporal structure.
    The $MT_{\tau}$- slice of $(\bar{\mathcal{D}}, \bar{\tau})$ is the time slice $(\bar{\mathcal{D}}', \bar{\tau}')$ of $(\bar{\mathcal{D}}, \bar{\tau})$, where $s:[0,l) \to \mathbb{N}$ is the function $s(i') = i' + c, l = |\{i \in \mathbb{N} \mid \tau_i \in T\}|$, and $c = \min\{i \in \mathbb{N} \mid \tau_i \in T \}$. 
    We also require that $\tau_l' \not\in T$ and 
    % TODO figure out what the reference time stamp is here
    \begin{equation*}
        \mathcal{D}'_{i'} = \filter(M, \mathcal{D}_{s(i')}, \tau_{s(i')}, \tau),
    \end{equation*}
    for all $i' \in [0,l)$.
\end{definition}


\begin{definition}
    \label{def:overlapping-ext}
    Let $I \subseteq \mathbb{Z}$ be an interval, and $M$ be a masked predicate map where for all $(m, J) \in M$, $J \subseteq I$.
    Let $c,i \in \mathbb{N}$.
    The temporal structures $(\bar{\mathcal{D}}, \bar{\tau})$ and $(\bar{\mathcal{D}}', \bar{\tau}')$ are $(M,I,c,i)$-overlapping if the following conditions hold:
    \renewcommand{\labelenumi}{\arabic{enumi}.}
    \begin{enumerate}
        \item 
            $j \geq c$,
            $\tau_j = \tau'_{j-c}$,
            and
            \begin{equation*}
                \filter(M, \mathcal{D}_j, \tau_j, \tau_i)
                = \filter(M, \mathcal{D}'_{j-c}, \tau'_{j-c}, \tau_i)
            \end{equation*}
            for all $j \in \mathbb{N}$ with $\tau_j - \tau_i \in I$,
        \item
            $\tau_{j'+c} = \tau'_{j'}$,
            and
            \begin{equation*}
                \filter(M, \mathcal{D}_{j'+c}, \tau_{j'+c}, \tau_i)
                = \filter(M, \mathcal{D}'_{j'}, \tau'_{j'}, \tau_i)
            \end{equation*}
            for all $j' \in \mathbb{N}$ with $\tau'_{j'} - \tau_i \in I$,
    \end{enumerate}
\end{definition}
% Like with Definition \ref{def:overlapping}, two traces are $(M,T,c,i)$-overlapping if their \textit{filtered} time points are the same on an interval of time stamps.
% TODO rephrase this previous sentence to more accurately describe (MTci)-overlapping

The next lemma establishes that a trace and its $MT_{\tau_i}$- slice are $(M,T,c,i)$- overlapping.
It is analogous to Lemma A.2 in Basin et al. \cite{Basin2016}.
% It is partially analogous to Lemma A.2 in Basin et al. \cite{Basin2016}.
% It however makes a less general statement about $MT_{\tau_i}$- slices and $(M,T,c,i)$- overlapping than Lemma A.2 makes about $T$- slices and $(I,c,i)$- overlapping.

\begin{lemma}
    \label{lem:mt-slice-overlap}
    Let $T \subseteq \mathbb{N}$ and $I \subseteq \mathbb{Z}$ be intervals, 
    $(\bar{\mathcal{D}}, \bar{\tau})$ a temporal structure, and $(\bar{\mathcal{D}}', \bar{\tau}')$ a $(T \oplusext M, T \oplus I)_{\tau_i}$- slice of $(\bar{\mathcal{D}}, \bar{\tau})$.
    The temporal structures $(\bar{\mathcal{D}}, \bar{\tau})$ and $(\bar{\mathcal{D}}', \bar{\tau}')$ are $(M,I,c,i)$- overlapping, for all $i \in \mathbb{N}$ with $\tau_i \in T$, where $c \in \mathbb{N}$ is the value in Definition \ref{def:overlapping-ext} used by the function $s$ with respect to $(\bar{\mathcal{D}}, \bar{\tau})$ and its slice $(\bar{\mathcal{D}}', \bar{\tau}')$.
\end{lemma}
\textit{Proof}
% TODO change proof to correspond to updated lemma
This proof follows the same structure as the proof to Lemma A.2 in Basin et al. \cite{Basin2016} and the first few steps are the same.
The precondition in Definition \ref{def:overlapping-ext}, that every interval in $M$ is contained in the interval $T$, is satisfied, as it is also a precondition for Lemma \ref{lem:mt-slice-overlap}.
First we show condition 1 in Definition \ref{def:overlapping-ext}.
For all $j \in \mathbb{N}$ with $\tau_j - \tau_i \in T$, it follows from $\tau_i \in I$ that $\tau_j \in T \oplus I$.
From $c = \min\{k \in \mathbb{N} \mid \tau_k \in T \}$ we get $j \geq c$.
Even more precisely  for all $j \in \mathbb{N}$ with $\tau_j \in T \oplus I$, we get that $j \in [c, l+c)$.
% As a direct consequence we get $j \geq c$.
From Definition \ref{def:mt-slice} it follows that
\begin{align*}
    \mathcal{D}'_{j-c}
    & = \filter(T \oplusext M, \mathcal{D}_{s(j-c)}, \tau_{s(j-c)}, \tau_i) \\
    & = \filter(T \oplusext M, \mathcal{D}_{j-c+c}, \tau_{j-c+c}, \tau_i) \\
    & = \filter(T \oplusext M, \mathcal{D}_{j}, \tau_{j}, \tau_i), 
\end{align*}
for all $j \in \mathbb{N}$ with $\tau_j - \tau_i \in T \oplus I$.
With the help of Lemma \ref{lem:oplusext-zero} and Definition \ref{def:filter} (shifting all intervals in a map, shifts all time stamps that fall into these intervals the same way) we can convince ourselves that the above fact implies
\begin{align*}
    \mathcal{D}'_{j-c}
    & = \filter((-T \oplus T) \oplusext M, \mathcal{D}_{s(j-c)}, \tau_{s(j-c)}, \tau_i) \\
    & = \filter(\{0\} \oplusext M, \mathcal{D}_{j-c+c}, \tau_{j-c+c}, \tau_i) \\
    & = \filter(M, \mathcal{D}_{j}, \tau_{j}, \tau_i), 
\end{align*}
for all $j \in \mathbb{N}$ with $\tau_j - \tau_i \in (-T \oplus T) \oplus I = I$.
Thus we get
\begin{align*}
    \mathcal{D}'_{j-c}
    & = \filter(M, \mathcal{D}_{j}, \tau_{j}, \tau_i), 
\end{align*}
for all $i \in \mathbb{N}$ with $\tau_i \in T$, and $j \in \mathbb{N}$ with $\tau_j - \tau_i \in I$.
From Lemma \ref{lem:double-filter} it follows that
\begin{align*}
    \mathcal{D}'_{j-c}
    &= \filter(M, \mathcal{D}_j, \tau_j, \tau_i) \\
    &= \filter(M, \filter(M, \mathcal{D}_j, \tau_j, \tau_i), \tau_j, \tau_i) \\
    &= \filter(M, \mathcal{D}'_{j-c}, \tau_j, \tau_i) \\
    &= \filter(M, \mathcal{D}'_{j-c}, \tau'_{j-c}, \tau_i) \\
\end{align*}
for all $j \in \mathbb{N}$.
And with that we have shown everything for Condition 1 in Definition \ref{def:overlapping-ext}.

For Condition 2 we can deduce that for all $i \in \mathbb{N}$ with $\tau_i \in T$ and for all $j' \in \mathbb{N}$ with $\tau'_{j'} - \tau_i \in I$, $\tau'_{j'} \in T \oplus I$.
And because $\tau'_l \not\in T \oplus I$, $j' \in [0,l)$.
Hence $\tau_{j'+c} = \tau_{s(j')} = \tau'_{j'}$.
From Definition \ref{def:mt-slice} it follows that
\begin{align*}
    \mathcal{D}'_{j'}
    & = \filter(T \oplusext M, \mathcal{D}_{s(j')}, \tau_{s(j')}, \tau_i) \\
    & = \filter(T \oplusext M, \mathcal{D}_{j'+c}, \tau_{j'+c}, \tau_i) \\
\end{align*}
for all $j' \in \mathbb{N}$ with $\tau_j' - \tau_i \in T \oplus I$.
Again with the help of Lemma \ref{lem:oplusext-zero} and Definition \ref{def:filter} we can convince ourselves that the above fact implies
\begin{align*}
    \mathcal{D}'_{j'}
    & = \filter((-T \oplus T) \oplusext M, \mathcal{D}_{s(j')}, \tau_{s(j')}, \tau_i) \\
    & = \filter(\{0\} \oplusext M, \mathcal{D}_{j'+c}, \tau_{j'+c}, \tau_i) \\
    & = \filter(M, \mathcal{D}_{j'+c}, \tau_{j'+c}, \tau_i), 
\end{align*}
for all $j' \in \mathbb{N}$ with $\tau_j' - \tau_i \in (-T \oplus T) \oplus I = I$.
Thus we get
\begin{align*}
    \mathcal{D}'_{j'}
    & = \filter(M, \mathcal{D}_{j'+c}, \tau_{j+c}, \tau_i), 
\end{align*}
And like with condition 1, from Lemma \ref{lem:double-filter} it follows that
\begin{align*}
    \mathcal{D}'_{j'}
    & = \filter(M, \mathcal{D}_{j'+c}, \tau_{j'+c}, \tau_i)\\
    & = \filter(M, \mathcal{D}_{j'+c}, \tau_{j'+c}, \tau_i)\\ 
    & = \filter(M, \filter(M, \mathcal{D}_{j'+c}, \tau_{j'+c}, \tau_i), \tau_{j'+c}, \tau_i) \\
    & = \filter(M, \mathcal{D}'_{j'}, \tau_{j'+c}, \tau_i) \\
    & = \filter(M, \mathcal{D}'_{j'}, \tau'_{j'}, \tau_i),\\
\end{align*}
for all $j' \in \mathbb{N}$.
And with that we have shown everything for Condition 2 in Definition \ref{def:overlapping-ext}.


Analogous to Lemma A.3 in Basin et al. \cite{Basin2016} the following Lemma states that any overlapping traces also overlap on parts of the overlapping section.

\begin{lemma}
    \label{lem:eri-overlap-transitivity}
    Let $(\bar{\mathcal{D}}, \bar{\tau})$
        and
        $(\bar{\mathcal{D}}', \bar{\tau}')$ be two temporal structures that are $(M,I,c,i)$-overlapping,
            for some masked predicate map $M$,
            $I \subseteq \mathbb{Z}$,
            $c \in \mathbb{N}$,
            and $i \in \mathbb{Z}$.
    Then $(\bar{\mathcal{D}}, \bar{\tau})$ and $(\bar{\mathcal{D}}', \bar{\tau}')$
    are $(N, K, c, k)$-overlapping,
        for each $k \in \mathbb{N}$
        with $\tau_k - \tau_i \in I$, $K \subseteq \{ \tau_i - \tau_k \} \oplus I$,
        and $N \subseteqmap \{ \tau_i - \tau_k \} \oplusext M$.
\end{lemma}
\textit{Proof} 
This proof follows a similar structure as the proof of Lemma A.3 in Basin et al. \cite{Basin2016}.
We first show that Condition 1 in Definition \ref{def:overlapping-ext} is satisfied, i.e.
for all $j \in \mathbb{N}$ with $\tau_j - \tau_k \in K$,
the following three things hold:
$j \geq c$,
$\tau_j = \tau'_{j-c}$,
and
$ \filter(N, \mathcal{D}_j, \tau_j, \tau_k) = \filter(N, \mathcal{D}'_{j-c}, \tau'_{j-c}, \tau_k)$.

For all $j \in \mathbb{N}$ with $\tau_j - \tau_k \in K$,
    it follows that $\tau_j - \tau_k + \tau_k - \tau_i \in \{\tau_k - \tau_i \} \oplus K$,
    it then follows that $\tau_j - \tau_i \in \{\tau_k - \tau_i\} \oplus K$. 
From $K \subseteq \{ \tau_i - \tau_k \} \oplus I$,
    we get $\{\tau_k - \tau_i\} \oplus K \subseteq \{\tau_k - \tau_i \} \oplus \{\tau_i - \tau_k \} \oplus I = I$,
    or in short $\tau_j - \tau_i \in I$.
And as we know that $(\bar{\mathcal{D}}, \bar{\tau})$
    and $(\bar{\mathcal{D}}', \bar{\tau}')$ are $(M,I,c,i)$-overlapping,
    it follows from Condition 1 in Definition \ref{def:overlapping-ext} that
    for all $j \in \mathbb{N}$ with $\tau_j - \tau_k \in K$,
    $j \geq c$, $\tau_j = \tau'_{j-c}$,
    and $\filter(M, \mathcal{D}_j, \tau_j, \tau_i) = \filter(M, \mathcal{D}'_{j-c}, \tau'_{j-c}, \tau_i)$.

For any interpretation $x \in r^{\filter(N, \mathcal{D}_j, \tau_j, \tau_k)}$, it follows that there exists a mask $m$ for $r$ that matches $x$ and has a relative interval $T_{rm}$ with $\tau_j - \tau_k \in T_{rm}$.
Since $N \subseteqmap \{ \tau_i - \tau_k \} \oplusext M$, there must exist a mask $m'$ for $r$ with $m' \nomoreprecise m$ and a relative interval $T_{rm'}$, for which $\tau_j - \tau_k \in \{ \tau_i - \tau_k \} \oplus T_{rm'}$.
It follows that $\tau_k - \tau_i + \tau_j - \tau_k \in \{ \tau_k - \tau_i\} \oplus \{ \tau_i - \tau_k\} \oplus T_{rm'}$.
And hence $\tau_j - \tau_i \in T_{rm'}$.
From
\begin{equation*}
    \filter(M, \mathcal{D}_j, \tau_j, \tau_i)
    = \filter(M, \mathcal{D}'_{j-c}, \tau'_{j-c}, \tau_i),
\end{equation*}
for all $j \in \mathbb{N}$ with $\tau_j - \tau_i \in I$ and $T_{rm'} \subseteq I$, it follows that 
$\tau'_{j-c} - \tau_i \in T_{rm'}$.

\begin{align*}
    \filter(\{ \tau_k - \tau_i \} \oplusext M, \mathcal{D}_j, \tau_j, \tau_k - \tau_i + \tau_i)
    &= \filter(\{ \tau_k - \tau_i \} \oplusext M, \mathcal{D}'_{j'}, \tau'_{j'}, \tau_k - \tau_i + \tau_i), 
\end{align*}
for all $j \in \mathbb{N}$ with $\tau_j - \tau_k \in K$.
We can convince ourselves that this holds, by looking at Definitions \ref{def:e-rel-int-ops} and \ref{def:filter}.
First $\oplusext$ shifts all intervals in a map.
Then $filter$ checks for any interpretation of a predicate if it has a matching mask in the masked predicate map and if a time stamp is in the relative interval of that mask.
It is now easy to see that, if the original time stamp was in the original interval, the shifted time stamp is also in the shifted interval.
And hence,
\begin{align*}
    \filter(\{ \tau_k - \tau_i \} \oplusext M, \mathcal{D}_j, \tau_j, \tau_k)
    &= \filter(\{ \tau_k - \tau_i \} \oplusext M, \mathcal{D}'_{j'}, \tau'_{j'}, \tau_k) 
\end{align*}
for all $j \in \mathbb{N}$ with $\tau_j - \tau_k \in K$.
We know that $N \subseteqmap \{\tau_k - \tau_i \} \oplusext M$ for $\tau_k - \tau_i \in I$.


With this we are almost done.


% We are done with Condition 1, if we can show that
% \begin{equation*}
%     \filter(M, \mathcal{D}_j, \tau_j, \tau_i) = \filter(M, \mathcal{D}'_{j'}, \tau'_{j'}, \tau_i),
% \end{equation*}
% for all $j \in \mathbb{N}$ with $\tau_j - \tau_i \in I$ implies 
% \begin{equation*}
%     \filter(N, \mathcal{D}_j, \tau_j, \tau_k) 
%     = \filter(N, \mathcal{D}'_{j-c}, \tau'_{j-c}, \tau_k),  
% \end{equation*}
% for all $j \in \mathbb{N}$ with $\tau_j - \tau_k \in K$.
% % The equality of two filtered structures $\mathcal{D}_f$ and $\mathcal{D}'_f$ can be shown by showing that for all predicates $r$, $r^{\mathcal{D}_f} = r^{\mathcal{D}'_f}$.

% By Definition \ref{def:filter} an interpretation $x$ is in $r^{\filter(N, \mathcal{D}_j, \tau_j, \tau_k)}$ iff $\exists p,m,J . ((p,m), J) \in N \land p = r \land m \matches x \land \tau_i - \tau_k \in J$.
% From $N \subseteqmap \{ \tau_i - \tau_k \} \oplusext M$ it follows that for any $p,m,J$, with $((p,m), J) \in N$ there must exist $p',m',J'$,
%     with $((p',m'), J') \in \{\tau_i - \tau_k \} \oplusext M$,
%     $p=p'$,
%     $m' \nomoreprecise m$,
%     and $J \subseteq J'$.
% From the definition of $\oplusext$ it is apparent that
% $((p',m'), J') \in \{\tau_i - \tau_k \} \oplusext M$
% iff
% $((p',m'), \{\tau_k - \tau_i\} \oplus J') \in M$.
% Hence there is an interpretation in the filtration by $N$, if it is in the filtration with $M$ shifted by $\{\tau_i - \tau_k\}$.
% And from the fact that the two temporal structures are $(M,I,c,i)$- overlapping

%TODO
% \textit{Needs to be finished}

\dots




With this we get to the key Lemma that we rely on to make our optimization when selecting and querying a sub-trace from the database.
It is analogous to Lemma A.4 in Basin et al. \cite{Basin2016}.

\begin{lemma}
    \label{lem:eri-overlap}
    Let $\phi$ be a formula and $(\bar{\mathcal{D}}, \bar{\tau})$ and $(\bar{\mathcal{D}}', \bar{\tau}')$ temporal structures.
    If $(\bar{\mathcal{D}}, \bar{\tau})$ and $(\bar{\mathcal{D}}', \bar{\tau}')$ are $(\ERI(\phi), \RI(\phi), c, i)$-overlapping, for some $c$ and $i$, then for all valuations $v$, it holds that $(\bar{\mathcal{D}},\bar{\tau},v,i) \models \phi$ iff $(\bar{\mathcal{D}}', \bar{\tau}', v, i-c) \models \phi$.
\end{lemma}
This Lemma applies to MFOTL formulas.
It can be extended to MFODL as well, but it needs a mutually recursive extensions to include regular expressions.
We focus on the core MFOTL formulas.

\textit{Proof} This proof is in large parts analogous to the one to Lemma A.4 in Basin et al. \cite{Basin2016}, and we follow the same structure. First we note that for the same reason as in Lemma A.4 in Basin et al. \cite{Basin2016}, the lemma's statement is well-defined, as the definition of $\RI(\phi)$ did not change and still includes $0$.
We proof the lemma by structural induction on the formula $\phi$.
In this proof $S = (C, R, \iota)$ is a signature, with constants $C$, predicats $R$, and the arity function $\iota$.
Similarly $V$ is the set of variables.
We have the following cases.

\begin{itemize}
    \item 
        $t \approx t'$, where $t, t' \in V \cup C$. 
        The satisfaction of $t \approx t'$ depends only on the valuation $v$.
        Therefore it follows that $(\bar{\mathcal{D}},\bar{\tau},v,i) \models t \approx t'$ iff $(\bar{\mathcal{D}}',\bar{\tau}',v,i-c) \models t \approx t'$.
    \item 
        $t \prec t'$ and $t \preceq$ are analogous to the first one and their detailed proofs are omitted.
    \item 
        $r(\bar{t})$, where $t_1, \dots, t_{\iota(r)} \in V \cup C$. 
        Since $(\bar{\mathcal{D}},\bar{\tau})$ and $(\bar{\mathcal{D}}',\bar{\tau}')$ are $(\ERI(\phi), \RI(\phi), c, i)$-overlapping it follows from Condition 1 in Definition \ref{def:overlapping-ext} that 
        \begin{equation*}
            \filter(\ERI(\phi), \mathcal{D}_j, \tau_j, \tau_i)
            = \filter(\ERI(\phi), \mathcal{D}'_{j-c}, \tau'_{j-c}, \tau_i),
        \end{equation*}
        $\tau_j = \tau'_{j-c}$, and $j \geq c$
        for all $j \in \mathbb{N}$ with $\tau_j - \tau_i \in \RI(\phi)$.
        % We make a case distinction on whether $\tau_j - \tau_i \in \RI(\phi)$.
        % \begin{itemize}
            % \item $\tau_j - \tau_i \in \RI(\phi)$.
                % In this case,
                By Definition \ref{def:filter}, there exists a valuation $v(\bar{t})$ in $r^{\mathcal{D}_j}$ iff there exist $m,J$, with $((r,m),J) \in \ERI(\phi)$ with $\tau_j - \tau_i \in J$, and $m \matches v(\bar{t})$.
                And from the equality of the two filters given to us by Condition 1 of Definition \ref{def:overlapping-ext} this is the case iff there exist $m', J'$, with $((r,m'),J') \in \ERI(\phi)$ with $\tau'_{j-c} - \tau_i \in J'$, and $m' \matches v(\bar{t})$ for all valuations $v$.
                And further this last part holds iff $v(\bar{t}) \in r^{\mathcal{D}'_{j-c}}$.
                Thus, by Definition, \ref{def:mfotl-evaluation}
                    $(\bar{\mathcal{D}},\bar{\tau},v,i) \models r(\bar{t})$
                    iff
                    $(\bar{\mathcal{D}}',\bar{\tau}',v,i-c) \models r(\bar{t})$.
        %     \item $\tau_j - \tau_i \not\in \RI(\phi)$.
        %         In this case it follows from Definition \ref{def:mfotl-evaluation} that
        % \end{itemize}

    \item 
        $\neg \psi$. 
        $(\bar{\mathcal{D}}, \bar{\tau})$ and $(\bar{\mathcal{D}}', \bar{\tau}')$ are $(\ERI(\phi), \RI(\phi), c, i)$-overlapping.
        By definition $\ERI(\neg \psi) = \ERI(\psi)$ and $\RI(\neg \psi) = \RI(\psi)$.
        By the inductive hypothesis $(\bar{\mathcal{D}}, \bar{\tau}, v, i) \models \psi$ iff $(\bar{\mathcal{D}}', \bar{\tau}', v, i-c) \models \psi$ for all valuations $v$.
        Therefore $(\bar{\mathcal{D}}, \bar{\tau}, v, i) \models \neg \psi$ iff $(\bar{\mathcal{D}}', \bar{\tau}', v, i-c) \models \neg \psi$ for all valuations $v$.

    \item 
        $\psi \lor \chi$.
        $(\bar{\mathcal{D}}, \bar{\tau})$ and $(\bar{\mathcal{D}}', \bar{\tau}')$ are $(\ERI(\psi) \Cupmerge \ERI(\chi), \RI(\psi) \Cup \RI(\chi), c, i)$-overlapping.
        From $\ERI(\psi) \subseteqmap \ERI(\psi) \Cupmerge \ERI(\chi)$, $\ERI(\chi) \subseteqmap \ERI(\psi) \Cupmerge \ERI(\chi)$, $\RI(\psi) \subseteq \RI(\psi) \Cup \RI(\chi)$, and $\RI(\chi) \subseteq \RI(\psi) \Cup \RI(\chi)$,
            it follows from Lemma \ref{lem:eri-overlap-transitivity} that $(\bar{\mathcal{D}}, \bar{\tau})$ and $(\bar{\mathcal{D}}', \bar{\tau}')$ are $(\ERI(\psi), \RI(\psi), c, i)$- and $(\ERI(\chi), \RI(\chi), c, i)$-overlapping.
        By the inductive hypothesis, $(\bar{\mathcal{D}}, \bar{\tau}, v, i) \models \psi$ iff $(\bar{\mathcal{D}}', \bar{\tau}', v, i-c) \models \psi$ and $(\bar{\mathcal{D}}, \bar{\tau}, v, i) \models \chi$ iff $(\bar{\mathcal{D}}', \bar{\tau}', v, i-c) \models \chi$.
        Therefore $(\bar{\mathcal{D}}, \bar{\tau}, v, i) \models \psi \lor \chi$ iff $(\bar{\mathcal{D}}', \bar{\tau}', v, i-c) \models \psi \lor \chi$.
        
    \item 
        $\exists x. \psi$.
        From $\ERI(\exists x. \psi) = \ERI(\psi)$ and $\RI(\exists x. \psi) = \RI(\psi)$ it follows that $(\bar{\mathcal{D}}, \bar{\tau})$ and $(\bar{\mathcal{D}}', \bar{\tau}')$ are $(\ERI(\psi), \RI(\psi), c, i)$-overlapping.
        By the inductive hypothesis we have $(\bar{\mathcal{D}}, \bar{\tau}, v, i) \models \psi$ iff $(\bar{\mathcal{D}}', \bar{\tau}', v, i-c) \models \psi$ for all valuations $v$.
        Thus for all $d$, $(\bar{\mathcal{D}}, \bar{\tau}, v[x \mapsto d], i) \models \psi$ iff $(\bar{\mathcal{D}}', \bar{\tau}', v[x \mapsto d], i-c) \models \psi$.
        By Definition \ref{def:mfotl-evaluation} it follows directly that $(\bar{\mathcal{D}}, \bar{\tau}, v, i) \models \exists x. \psi$ iff $(\bar{\mathcal{D}}', \bar{\tau}', v, i-c) \models \exists x. \psi$ for all valuations $v$.

    \item
        $\Previous{[a,b)} \psi$.
        $(\bar{\mathcal{D}}, \bar{\tau})$ and $(\bar{\mathcal{D}}', \bar{\tau}')$ are $(\ERI(\Previous{[a,b)} \psi), \RI(\Previous{[a,b)} \psi), c, i)$- overlapping, where
        \begin{equation*}
            \ERI(\Previous{[a,b)} \psi) = (-b,0] \Cupext ((-b,-a] \oplusext \ERI(\psi))
        \end{equation*}
        and
        \begin{equation*}
            \RI(\Previous{[a,b)} \psi) = (-b,0] \Cup ((-b,-a] \oplus \RI(\psi))
        \end{equation*}
        From $0 \in \RI(\Previous{[a,b)} \psi)$ and Condition 1 in Definition \ref{def:overlapping-ext}, it follows that $\tau_i = \tau'_{i-c}$.

        We make a case split on the value of $i$. If $i=0$, then $(\bar{\mathcal{D}}, \bar{\tau}, v, i) \not\models \Previous{[a,b)} \psi$, for all valuations $v$.
        From $c \in \mathbb{N}, i-c \in \mathbb{N}$, and $i=0$, it follows that $i-c=0$.
        Since $i-c = 0 = i$ it is trivial that $(\bar{\mathcal{D}}', \bar{\tau}', v, i-c) \not\models \Previous{[a,b)} \psi$ for all valuations $v$.

        Next we consider $i > 0$ and make another case split on whether $\tau_i - \tau_{i-1} \in [a,b)$.
        \begin{itemize}
            \item 
                $\tau_i - \tau_{i-1} \in [a,b)$.
                Then $\tau_{i-1} - \tau_i \in \RI(\Previous{[a,b)} \psi)$ and from Condition 1 in Definition \ref{def:overlapping-ext} it follows that $i-1 \geq c$, $\tau_{i-1} = \tau'_{i-c-1}$, and hence $\tau'_{i-c} - \tau'_{i-c-1} \in [a,b)$.
                It further follows that
                \begin{align*}
                    \ERI(\psi) 
                    &\subseteqmap \{\tau_i - \tau_{i-1} \} \oplusext \{ \tau_{i-1} - \tau_i \} \oplusext \ERI(\psi) \\
                    &\subseteqmap \{\tau_i - \tau_{i-1} \} \oplusext (-b, -a] \oplusext \ERI(\psi) \\
                    &\subseteqmap \{\tau_i - \tau_{i-1} \} \oplusext \ERI(\Previous{[a,b)} \psi)
                \end{align*}
                and
                \begin{align*}
                    \RI(\psi) 
                    &\subseteq \{\tau_i - \tau_{i-1} \} \oplus \{ \tau_{i-1} - \tau_i \} \oplus \RI(\psi) \\
                    &\subseteq \{\tau_i - \tau_{i-1} \} \oplus (-b, -a] \oplus \RI(\psi) \\
                    &\subseteq \{\tau_i - \tau_{i-1} \} \oplus \RI(\Previous{[a,b)} \psi).
                \end{align*}
                Thus by Lemma \ref{lem:eri-overlap-transitivity} $(\bar{\mathcal{D}}, \bar{\tau})$ and $(\bar{\mathcal{D}}', \bar{\tau}')$ are $(\ERI(\psi), \RI(\psi), c, i-1)$- overlapping.
                By the induction hypothesis $(\bar{\mathcal{D}}, \bar{\tau}, v, i-1) \models \psi$ iff $(\bar{\mathcal{D}}', \bar{\tau}', v, i-c-1) \models \psi$ for all valuations $v$.
                Because $\tau_i = \tau'_{i-c}$ and $\tau_{i-1} = \tau'_{i-c-1}$, it follows that $(\bar{\mathcal{D}}, \bar{\tau}, v, i) \models \Previous{[a,b)} \psi$ iff $(\bar{\mathcal{D}}', \bar{\tau}', v, i-c) \models \Previous{[a,b)} \psi$ for all valuations $v$.
            \item 
                $\tau_i - \tau_{i-1} \not\in [a,b)$.
                Then $(\bar{\mathcal{D}}, \bar{\tau}, v, i) \not\models \Previous{[a,b)} \psi$, for all valuations $v$.
                From Condition 1 of Definition \ref{def:overlapping-ext} we still have $i \geq c$.
                We make a case distinction on whether $i=c$ or $i>c$.
                \begin{itemize}
                    \item
                        $i=c$.
                        In this case $i-c = 0$.
                        Therefore $(\bar{\mathcal{D}}, \bar{\tau}, v, i-c) \not\models \Previous{[a,b)} \psi$, for all valuations $v$.
                    \item
                        $i>c$.
                        We proof this case by contradiction.
                        Assume that $\tau'_{i-c} - \tau'_{i-c-1} = \tau'_{i-c} - \tau'_{i-c-1} \in [a,b)$.
                        From Condition 2 in Definition \ref{def:overlapping-ext} it follows that $\tau_{i-1} = \tau'_{i-c-1}$ and hence $\tau_i - \tau_{i-1} = \tau'_{i-c} - \tau'_{i-c-1} \in [a,b)$.
                        This contradicts $\tau_i - \tau_{i-1} \not\in [a,b)$, so it must be the case that $\tau'_{i-c} - \tau'_{i-c-1} \not\in [a,b)$.
                        It follows that $(\bar{\mathcal{D}}', \bar{\tau}', v, i-c) \not\models \Previous{[a,b)} \psi$, for all valuations $v$.
                \end{itemize}
        \end{itemize}
    \item
        $\Next{[a,b)}$.
        The "next" case is in large parts similar to the one for "previous".
        Like with Lemma A.4 \cite{Basin2016} it is simpler than "previous", because no special distinctions have to be made for $i=0$ and $i-c=0$.
    \item
        $\psi \Since{[a,b)} \chi$.
        Like the other cases this one is also very similar to the same case in the proof to Lemma A.4 \cite{Basin2016} and we follow the same structure.
        $(\bar{\mathcal{D}},\bar{\tau})$ and $(\bar{\mathcal{D}}',\bar{\tau}')$ are $(\ERI(\psi \Since{[a,b)} \chi), \RI(\psi \Since{[a,b)} \chi), v, i)$- overlapping, where
        \begin{align*}
            \ERI(\psi \Since{[a,b)} \chi)
            &= (-b,0] \Cupext ((-b,0] \oplusext \ERI(\psi)) \Cupmerge ((-b,-a] \oplusext \ERI(\chi))
        \end{align*}
        and
        \begin{align*}
            \RI(\psi \Since{[a,b)} \chi)
            &= (-b,0] \Cup ((-b,0] \oplus \RI(\psi)) \Cup ((-b,-a] \oplus \RI(\chi)).
        \end{align*}
        From $0 \in \RI(\psi \Since{[a,b)} \chi)$ and Condition 1 in Definition \ref{def:overlapping-ext} it follows that $\tau_i = \tau'_i-c$.

        First we establish the two following claims, by applying the inductive hypothesis to the two sub-formulas $\psi$ and $\chi$.
        \begin{enumerate}
            \item ($\chi$)
                For all $j \in \mathbb{N}$ with $j \leq i$ and $\tau_i - \tau_j \in [a,b)$, it holds that 
                \begin{align*}
                    \RI(\chi) 
                    &\subseteq \{ \tau_i - \tau_j \} \oplus \{ \tau_j - \tau_i\} \oplus \RI(\chi) \\
                    &\subseteq \{\tau_i - \tau_j \} \oplus (-b, -a] \oplus \RI(\chi) \\
                    &\subseteq \{ \tau_i - \tau_j \} \oplus \RI(\psi \Since{[a,b)} \chi)
                \end{align*}
                as well as
                \begin{align*}
                    \ERI(\chi) 
                    &\subseteqmap \{ \tau_i - \tau_j \} \oplusext \{ \tau_j - \tau_i\} \oplusext \ERI(\chi) \\
                    &\subseteqmap \{\tau_i - \tau_j \} \oplusext (-b, -a] \oplusext \ERI(\chi) \\
                    &\subseteqmap \{ \tau_i - \tau_j \} \oplusext \ERI(\psi \Since{[a,b)} \chi)
                \end{align*}
                and $j \geq c$.
                We can convince ourselves that the second step in this equation holds, by considering how the $\oplusext$ operator shifts and or resizes the relative intervals inside a masked predicate map.
                By Lemma \ref{lem:eri-overlap-transitivity} $(\bar{\mathcal{D}}, \bar{\tau})$ and $(\bar{\mathcal{D}}', \bar{\tau}')$ are $(\ERI(\chi), \RI(\chi), c, j)$- overlapping.
                It then follows from the inductive hypothesis that
                \begin{equation*}
                    (\bar{\mathcal{D}}, \bar{\tau}, v, j) \models \chi
                    \text{ iff }
                    (\bar{\mathcal{D}}', \bar{\tau}', v, j-c) \models \chi,
                \end{equation*}
                for all valuations $v$.
            \item ($\psi$)
                For all $k \in \mathbb{N}$ with $k \leq i$ and $\tau_i - \tau_k \in [0,b)$, it holds that
                \begin{align*}
                    \RI(\psi)
                    &\subseteq \{ \tau_i - \tau_k \} \oplus \{ \tau_k - \tau_i \} \oplus \RI(\psi) \\
                    &\subseteq \{ \tau_i - \tau_k \} \oplus (-b,0] \oplus \RI(\psi) \\
                    &\subseteq \{ \tau_i - \tau_k \} \oplus \RI(\psi \Since{[a,b)} \chi) \\
                \end{align*}
                as well as
                \begin{align*}
                    \ERI(\psi)
                    &\subseteqmap \{ \tau_i - \tau_k \} \oplusext \{ \tau_k - \tau_i \} \oplusext \ERI(\psi) \\
                    &\subseteqmap \{ \tau_i - \tau_k \} \oplusext (-b,0] \oplusext \ERI(\psi) \\
                    &\subseteqmap \{ \tau_i - \tau_k \} \oplusext \ERI(\psi \Since{[a,b)} \chi) \\
                \end{align*}
                and $k \geq c$.
                By Lemma \ref{lem:eri-overlap-transitivity} $(\bar{\mathcal{D}}, \bar{\tau})$ and $(\bar{\mathcal{D}}', \bar{\tau}')$ are $(\ERI(\psi), \RI(\psi), c, k)$- overlapping.
                It follows from the inductive hypothesis that
                \begin{equation*}
                    (\bar{\mathcal{D}}, \bar{\tau}, v, k) \models \psi
                    \text{ iff }
                    (\bar{\mathcal{D}}', \bar{\tau}', v, k-c) \models \psi,
                \end{equation*}
                for all valuations $v$.
                
        \end{enumerate}
        We show each direction of the claimed equivalence, $(\bar{\mathcal{D}}, \bar{\tau}, v, i) \models \psi \Since{[a,b)} \chi$ iff $(\bar{\mathcal{D}}', \bar{\tau}', v, i-c) \models \psi \Since{[a,b)} \chi$, for all valuations $v$, separately and start with the direction from left to right.
        If $(\bar{\mathcal{D}}, \bar{\tau}, v, i) \models \psi \Since{[a,b)} \chi$ then there is some $j \leq i$ with $\tau_i - \tau_j \in [a,b)$ such that $(\bar{\mathcal{D}}, \bar{\tau}, v, j) \models \chi$ and $(\bar{\mathcal{D}}, \bar{\tau}, v, k) \models \psi$, for all $k \in [j+1, i+1)$.

        From $\tau_i - \tau_j \in [a,b)$ it follows that $\tau_j - \tau_i \in \RI(\psi \Since{[a,b)} \chi)$ and from Condition 1 in Definition \ref{def:overlapping-ext} it follows that $j \geq c$ and $\tau_j = \tau'_{j-c}$.
        From Claim (i) and from $(\bar{\mathcal{D}}, \bar{\tau}, v, j) \models \chi$ it follows that $(\bar{\mathcal{D}}', \bar{\tau}', v, j-c) \models \chi$.

        For all $k' \in [j+1-c, i+1-c)$, it holds that $\tau'_{k'} - \tau'_{i-c} = \tau'_{k'} - \tau_i \in (-b,0]$ and hence $\tau'_{k'} - \tau_i \in \RI(\psi \Since{[a,b)} \chi)$.
        % TODO continue
        It follows from Condition 2 in Definition \ref{def:overlapping-ext} that $\tau_{k'+c} = \tau'_{k'}$.
        With $(\bar{\mathcal{D}}, \bar{\tau}, v, k' + c) \models \psi$ and Claim (ii) we get $(\bar{\mathcal{D}}', \bar{\tau}', v, k') \models \psi$.
        And with that $(\bar{\mathcal{D}}', \bar{\tau}', v, i-c) \models \psi \Since{[a,b)} \chi$.
        This concludes the proof in the first direction.

        Next we show the other direction of the equivalence, i.e. if $(\bar{\mathcal{D}}', \bar{\tau}', v, i-c) \models \psi \Since{[a,b)} \chi$ then $(\bar{\mathcal{D}}, \bar{\tau}, v, i) \models \psi \Since{[a,b)} \chi$.
        This is equivalent to $(\bar{\mathcal{D}}, \bar{\tau}, v, i) \not\models \psi \Since{[a,b)} \chi$ then $(\bar{\mathcal{D}}', \bar{\tau}', v, i-c) \not\models \psi \Since{[a,b)} \chi$.
        Like in the proof to Lemma A.4 \cite{Basin2016}, we show the second implication.
        From Definition \ref{def:mfotl-evaluation} it follows that there are two possibilities for $(\bar{\mathcal{D}}, \bar{\tau}, v, i) not\models \psi \Since{[a,b)} \chi$.
        The first is that for no $j \leq i$, $\tau_i - \tau_j \in [a,b)$ and $(\bar{\mathcal{D}}, \bar{\tau}, v, i) \models \chi$.
        Or written differently, for all $j \leq i$ with $\tau_i - \tau_j \in [a,b)$, $(\bar{\mathcal{D}}, \bar{\tau}, v, i) \not\models \chi$.
        From the Definition \ref{def:rel-int} it follows that for all $j' \leq i-c$ with $\tau'_{i-c} - \tau'_{j'} = \tau_i - \tau'_{j'} \in [a,b)$, $\tau'_{j'} - \tau_i \in \RI(\psi \Since{[a,b)} \chi)$.
        This is a useful property, because it means we can conclude from Condition 2 of Definition \ref{def:overlapping-ext} that $\tau_{j'+c} = \tau'_{j'}$ for all $j' \leq i-c$.
        And as in the proof to Lemma A.4 \cite{Basin2016} this means that there cannot be a time stamp within the interval $[a,b)$ in $(\bar{\mathcal{D}}', \bar{\tau}')$ that is not present in $(\bar{\mathcal{D}}, \bar{\tau})$.
        We are currently looking at the case where for all $j \leq i$, $(\bar{\mathcal{D}}, \bar{\tau}, v, j) \not\models \chi$.
        Thus also $(\bar{\mathcal{D}}, \bar{\tau}, v, j'+c) \not\models \chi$ for all $j' \leq i-c$.
        Combining this with Claim (i) from above we get that $(\bar{\mathcal{D}}', \bar{\tau}', v, j') \not\models \chi$.
        And by Definition \ref{def:mfotl-evaluation} it follows that $(\bar{\mathcal{D}}', \bar{\tau}', v, i-c) \not\models \psi \Since{[a,b)} \chi$.

        The second possibility for $(\bar{\mathcal{D}}, \bar{\tau}, v, i) not\models \psi \Since{[a,b)} \chi$ is, that for all $j \leq i$ with $\tau_i - \tau_j \in [a,b)$ and $(\bar{\mathcal{D}}, \bar{\tau}, v, j) \models \chi$ there is some $k \in \mathbb{N}$ with $j < k \leq i$, $(\bar{\mathcal{D}}, \bar{\tau}, v, k) \not\models \psi$.
        It follows from this that for every $j' \in \mathbb{N}$ with $j' \leq i-c$, $\tau'_{i-c} - \tau'_{j'} \in [a,b)$, and $(\bar{\mathcal{D}}', \bar{\tau}', v, j') \models \chi$, there exists a $j \in \mathbb{N}$ with $j = j' + c$.
        It follows from $\tau'_{i-c} - \tau'_{j'} \in [a,b)$ and $\tau'_{i-c} = \tau_i$, that $\tau'_{j'} - \tau_i \in (-b,-a]$ and further with the help of Definition \ref{def:rel-int} it follows that $\tau'_{j'} - \tau_i \in \RI(\psi \Since{[a,b)} \chi)$.
        From this we recall Condition 2 from Definition \ref{def:overlapping-ext} and see that $\tau_j = \tau'_{j-c} = \tau'{j'}$.
        Since $j' \leq i-c$ and $j = j'+c$ it follows that $j \leq i$.
        By having shown that $\tau'_{j'} = \tau_j$ and $j \leq i$, we can now make use of Claim (i) from above again and combine it with $(\bar{\mathcal{D}}, \bar{\tau}, v, j) \models \chi$.
        It follows that $(\bar{\mathcal{D}}', \bar{\tau}', v, j'+c) \models \chi$.
        From $\tau'_{j'} = \tau_{j+c} = \tau_j$ it follows that $\tau_k = \tau'_{k-c}$
        Going back around we now have that there is a $k \in \mathbb{N}$ with $ j' < k-c \leq i-c$, for which $(\bar{\mathcal{D}}', \bar{\tau}', v, k-c) \not\models \chi$.
        



        %TODO

        \dots


    \item
        $\psi \Until{[a,b)} \chi$.
        The "Until" case is analogous to the "Since" case.
\end{itemize}


For the MFODL cases, just like for "normal" regular intervals an additional lemma is needed, the proof of needs to be done over both of them, as they are mutually recursive.
We state this additional lemma here, but forgo the proof for time reasons.

\begin{lemma}
    \label{lem:eri-completness-reg}
    Let $\rho$ be a regular expression and $(\bar{\mathcal{D}}, \bar{\tau}), (\bar{\mathcal{D}}', \bar{\tau}')$ temporal structures.
    If $(\bar{\mathcal{D}}, \bar{\tau})$ and $(\bar{\mathcal{D}}', \bar{\tau}')$ are $(\ERIr(\rho),\RIr(\rho),c,i)$-overlapping, for some $c$ and $i$, then for all valuations $v$ and $j \in \mathbb{N}$, it holds that $(\bar{\mathcal{D}}, \bar{\tau},v,i,j) \modelsreg \rho$ iff $(\bar{\mathcal{D}}',\bar{\tau}',v,i-c,j-c) \modelsreg \rho$.
\end{lemma}



%%% Justification/ intuition for also getting all time points inside the entire regular interval
To get a correct sub trace $\sigma'$ we cannot simply extract the sub trace per predicate and mask.
There are possibly time points and time stamps in the original trace that have no predicate attached to them.
Therefore we additionally need to extract all such time stamps that fall into the regular relative interval of the formula.
Empty time points or time stamps can influence the truth value of a formula and can therefore not be omitted.

The first case of atomic formulas that are not a predicate is trivial.
They do not depend on any events that may or may not be present in a log and can be evaluated as is.
Next a simple predicate only depends on the current time point with time stamp $0$.
For the unary and binary first order logic formulas (negation, quantification, and, or) the extended relative interval is the special union of relative intervals of the sub formula(s).
Special union referring to the operator that keeps the intervals as intervals.