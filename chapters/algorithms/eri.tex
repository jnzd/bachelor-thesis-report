\section{Relative Interval Extension}

This idea of relative intervals can already filter an existing trace down to a much smaller one by removing events that are unnecessary for the evaluation of a given policy.
We expand on this by creating and using a data structure that allows us to select an even smaller sub trace with the same effect of not changing the truth value of the policy.

First we move from one relative interval for an entire policy to one relative interval per predicate occurring in a policy.
We break this down further.
Every predicate comes with a number of attributes as defined in the signature.
Some attributes are potentially constant.

Looking back at our policy from Figure \ref{fig:example-policy}, \texttt{"advertising"} is one such constant attribute in the predicate \texttt{loc\_accessed}.
This means any occurrence of the predicate \texttt{loc\_accessed} where the second attribute is not \texttt{"advertising"}, has no influence on our policy and is therefore not needed in a potential sub trace.
We check every predicate in our policy for constant attributes.
Then we take the set of different arrangements of constant and variable attributes per predicate.
We define a structure that captures constant and variable attributes of a predicate.
\begin{definition}
    \label{def:mask}
    Let $S=(C,R,\iota)$ be a signature and $r \in R$ a predicate with arity $\iota(r)$.
    A mask for the predicate $r$ is a tuple $m=(m_1, \dots, m_{\iota(r)})$ with $m_1, \dots, m_{\iota(r)} \in C \cup \{v\}$ and $v \not\in V \cup C$.
\end{definition}
$v$ is a placeholder value denoting attributes in the mask that have a non-constant value.
Each mask has its own relative interval.
For our example the masks with their corresponding relative intervals can be seen in Figure \ref{fig:example-ext-intervals}.

\begin{figure}[H]
    \label{fig:example-ext-intervals}
\begin{verbatim}
loc_accessed(*,"advertising") -> [0,0]
perm_granted(*) -> (*,0]
perm_revoked(*) -> (*,0]
\end{verbatim}
    \caption{Extended Relative Intervals of Example Policy}
\end{figure}

A \texttt{*} (asterisk) in the attributes denotes a variable value, $v$ in Definition \ref{def:mask}.
In larger formulas there can be multiple different masks per predicate.
Let $\mathcal{M}(r)$ be the set of possible masks for a predicate $r$.

\begin{definition}
    \label{def:mask-precision-comparison}
    Let $m_1=(m_{11}, \dots, m_{1\iota(r)})$ and $m_2=(m_{21}, \dots, m_{2\iota(r)})$ be two masks for the same predicate $r$.
    We say $m_1$ is at most as precise as $m_2$ if for all $j \in [1,\iota(r)]$, $m_{1j} = v$ or $m_{1j} = m_{2j}$.
    We write $m_1 \preceq_{\text{mask\_precision}} m_2$ for "$m_1$ is at most as precise as $m_2$.
\end{definition}

We use a map data structure to store the predicates, masks, and their respective relative intervals.

\begin{definition}
    \label{def:map}
    Let $S=(C,R,\iota)$ be a signature and let $\mathcal{M} : R \to \{(C \cup \{v\})^*\}$ be the function $\mathcal{M}(r) = M$ that gives the set of all possible masks $M$ for a predicate $r$.
    An \textbf{masked predicate map} is a set $\{(k,i)\}$ where $k = (r,m)$ with $r \in R$ and $m \in \mathcal{M}(r)$ and $i \subseteq \mathbb{Z}$ is an interval over $\mathbb{Z}$. 
\end{definition}


On this data structure we define the operators $\Cupmerge$, $\Cupext$ and $\oplusext$.

\begin{definition}
    \label{def:e-rel-int-ops}
    Let $M$ and $N$ be two masked predicate maps and $T$ a positive interval, then 
    \begin{align*}
        M \Cupmerge N = 
            & \{ p(l) \rightarrow (I \Cup J) \mid 
                p(l) \rightarrow I \in m \text{ and } 
                p(l) \rightarrow J \in n \} \\
            & \cup \{p(l) \rightarrow I \mid  
                (p(l) \rightarrow I \in m \text{ and }
                p(l) \in \keys(M) \setminus \keys(N)) \} \\
            & \cup \{p(l) \rightarrow I \mid  
                (p(l) \rightarrow I \in n \text{ and }
                p(l) \in \keys(N) \setminus \keys(M))
                \}        
                \\
        T \Cupext M = 
            & \{ p(l) \rightarrow (T \Cup I) \mid 
                p(l) \rightarrow I \in M \} \\
        T \oplusext M = 
            & \{ p(l) \rightarrow (T \oplus I) \mid 
                p(l) \rightarrow I \in M \} \\
    \end{align*}
\end{definition}

The notation $p(l) \rightarrow I$ denotes an element in an masked predicate map.
It is equivalend to the notation $((p,l), I)$, but it better shows how we are working with a map.
The $\keys$ operator gives all the first elements, the predicate name and mask tuples, in a masked predicate map.
With the help of the operators $\Cupmerge$, $\Cupext$ and $\oplusext$ we now give a recursive definition for our extension of relative intervals.

We define a relation on masked predicat maps that is analogous to the subset-equals relation for intervals and illustrates that one map is contained in another.

\begin{definition}
    \label{def:subseteqmap}
    Let $M$ and $N$ be two masked predicate maps.
    \begin{equation*}
        N \subseteqmap M
    \end{equation*}
    if for all $((p,l) \to I) \in N$ there exists $((p',l') \to I') \in M$ with
    \begin{equation*}
        (p = p') \land (l' \preceq_{\text{mask\_precision}} l) \land (I \subseteq I')
    \end{equation*}
\end{definition}

\begin{lemma}
    \label{lem:map-union-subsets}
    Let $M$ and $N$ be masked predicate maps, then
    \begin{equation*}
        M \subseteqmap M \Cupmerge N
    \end{equation*}
    and
    \begin{equation*}
        N \subseteqmap M \Cupmerge N.
    \end{equation*}
\end{lemma}
\textit{Proof}
Since the definition of $M \Cupmerge N$ is symmetric, it suffices to show on of the two relations in the lemma.
From Definition \ref{def:e-rel-int-ops} it follows that any key consisting of a predicate and a mask is also in $M \Cupmerge N$.
Since the mask is there unaltered, it satisfies the condition of being no more precise than the one in $M$.
The condition for $(I \subseteq I')$ is apparant from the definition as well, no interval is discarded and all intervals are either enlarged by the special union of intervals or kept the same.


\begin{definition}
    \label{def:e-rel-int}
    The extended relative interval of the formula $\varphi$, $\ERI(\varphi)$ is defined recursively over the formula structure:
    \begin{equation*}
        \ERI(\varphi) = \\
        \begin{cases}
            \{\} 
                & \text{if $\varphi$ is an an atomic formula} \\ &\text{and not a predicate,} \\ 
            \{p(m) \rightarrow [0,0]\} 
                & \text{if $\varphi$ is a predicate with name } \\ &\text{$p$ and mask $m$,} \\
            \ERI(\psi) 
                & \text{if $\varphi$ is of the form $\neg \psi, \exists x.\psi$,} \\
                & \text{or $\forall x.\psi$,} \\
            \ERI(\psi) \Cupmerge \ERI(\chi) 
                & \text{if $\varphi$ is of the form $\psi \lor \chi$,} \\ & \text{or $\psi \land \chi$,} \\
            (-b,0] \Cupext ((-b,-a] \oplusext \ERI(\psi)) 
                & \text{if $\varphi$ is of the form $\Previous{[a,b)}\psi$,} \\
            [0,b) \Cupext ([a,b) \oplusext \ERI(\psi)) 
                & \text{if $\varphi$ is of the form $\Next{[a,b)}$,}\\
            (-b,0] \Cupext ((-b,0] \oplusext \ERI(\psi)) \Cupmerge ((-b,-a] \oplusext \ERI(\chi)) 
                & \text{if $\varphi$ is of the form $\psi \Since{[a,b)} \chi$,} \\
            [0,b) \Cupext ([0,b) \oplusext \ERI(\psi)) \Cupmerge ([a,b) \oplusext \ERI(\chi)) 
                & \text{if $\varphi$ is of the form $\psi \Until{[a,b)} \chi$,} \\
            [0,b) \Cupext ([0,b) \oplusext \ERIr(\psi)) 
                & \text{if $\varphi$ is of the form $\Fregex{[a,b)} \psi$, and}\\
            (-b,0] \Cupext ((-b,0] \oplusext \ERIr(\psi)) 
                & \text{if $\varphi$ is of the form $\Pregex{[a,b)} \psi$.}\\
        \end{cases}
    \end{equation*}
\end{definition}

And for regular expressions we define 
\begin{definition}
    \label{def:e-rel-int-reg}
    The extended relative interval of the regular expression $\rho$, $\ERIr(\rho)$ is defined recursively over the structure of the regular expression:
    \begin{equation*}
        \ERIr(\rho) =
        \begin{cases}
            \{\} & \text{if $\rho$ is of the form $\star^k$,} \\
            \ERI(\varphi) & \text{if $\rho$ is of the form $\varphi ?$,} \\
            \ERIr(\sigma) \Cupmerge \ERIr(\tau) & \text{if $\rho$ is of the form $\sigma + \tau$ or $\sigma \cdot \tau$, and} \\
            \ERIr(\sigma) & \text{if $\rho$ is of the form $\sigma^*$.}
        \end{cases}
    \end{equation*}
\end{definition}

% \begin{definition}
%     \label{def:e-filter}
%     We call fine-grained filtering of a trace $\sigma=((\tau_i,D_i))_i$ according to the <name of your datastructure> X the trace 
    
%     \begin{equation*}
%         \filter'(\sigma, X, j) = ....
%     \end{equation*}
% \end{definition} 

% \begin{theorem}
%     \label{theo:e-filter-correctness}
%     For all $\sigma, v, i$:
%     \begin{equation*}
%         \sigma, v, i \models \varphi 
%         \Leftrightarrow \filter'(\sigma, \ERI(\phi), i), v, i \models \varphi
%     \end{equation*}
% \end{theorem} 

\subsection{Correctness}
Analogously to regular intervals we want to use the extended relative intervals to extract (slice) a sub trace from a larger trace for a given formula that has the property that the sub trace is sufficient to evaluate the formula.
We first need a few definitions and lemmata before we can proof this property.
\begin{definition}
    \label{def:matching-predicate}
    Let $\mathcal{D}$ be a structure over the signature $S = (C,R,\iota)$, $r^{\mathcal{D}} \in |\mathcal{D}|$ the set of interprations for the predicate $r \in R$, and $m$ a mask for $r$.
    The arity of $r$, $\iota(r)$ is the same as the length of the mask $m$, $|m|$.
    The mask $m = (m_1, \dots, m_{\iota(r)})$ matches the interpretation $k = (k_1, \dots, k_{\iota(r)}) \in r^{\mathcal{D}}$ if for all $i \in [1,\iota(r)]$ either $m_i = v$ or $m_i = k_i$, where $v$ is again the place holder value for variable values.
\end{definition}
\begin{definition}
    \label{def:filter}
    Let $M$ be a masked predicate map $\tau \in \mathbb{N}$ a time stamp and $\mathcal{D}$ a structure over the signature $S = (C,R,\iota)$.
    Let $\mathbb{D}$ be the set of all structures over the signature $S$.
    $\filter(M, \tau,\mathcal{D}) : \mathbb{D} \to \mathbb{D}$ is a function only keeping interprations in $r^{\mathcal{D}} \in |\mathcal{D}|$ that have a matching predicate mask $m$ and relative interval $I$, $(m,I) \in M$ with $\tau \in I$ as well as constant interpretations $c^{\mathcal{D}} \in |\mathcal{D}|$.
    Concretely $r^{\filter(M, \tau, \mathcal{D})} = \{v(\bar{t}) \in r^{\mathcal{D}} \mid \exists ((p,m), I) \in M. p = r \land m \text{ "matches" } \bar{t} \land \tau \in I \}$.
\end{definition}
The following lemma states that filtering a structure twice has the same effect as filtering once.
\begin{lemma}
    \label{lem:double-filter}
    Let $\mathcal{D}$ be a structure, let $M$ be a masked predicate map and let $\tau \in \mathbb{N}$ be a time stamp.
    Then $\filter(M, \tau ,\filter(M, \tau, \mathcal{D})) = \filter(M, \tau, \mathcal{D})$.
\end{lemma}
\textit{Proof} From Definition \ref{def:filter} we have that the constant interpretations remain untouched by the filter operation.
It remains to show that the left hand side (LHS) and the right hand side (RHS) in Lemma \ref{lem:double-filter} have the same interpretations for predicates.
The RHS only contains interpretations for predicates from $\mathcal{D}$ that have a matching mask for which its relative interval contains $\tau$.
The LHS contains interpretations fullfilling the same constraint, but they are "sampled" from the structure $\filter(M,\tau,\mathcal{D})$ instead of $\mathcal{D}$ directly.
But we already established that all predicate interpretations in the filtered structure (RHS) do fullfill the condition that they have a matching mask in $M$ whose relative interval contains $\tau$.
Therefore the LHS and RHS are equivalent and the Lemma holds.


% \begin{definition}
%     \label{def:filter-trace}
%     Let $(\bar{\mathcal{D}}, \bar{\tau}) = ((\mathcal{D}_1, \dots, \mathcal{D}_n), (\tau_1, \dots, \tau_n))$ be a temporal structure and let $M$ be a masked predicate map.
%     Then
%     \begin{equation*}
%         \filtertrace(M, \bar{\tau}, \bar{\mathcal{D}}) = ((\filter(M, \tau_1, \mathcal{D}_1), \dots, \filter(M, \tau_n, \mathcal{D}_n)), (\tau_1, \dots, \tau_n))   
%     \end{equation*}
%     is the temporal structure with all its structures filtered by $M$.
% \end{definition}

% \begin{lemma}
%     \label{lem:double-filter-trace}
%     Let $(\bar{\mathcal{D}}, \bar{\tau})$ be a temporal structure and let $M$ be a masked predicate map.
%     Then 
%     \begin{equation*}
%         \filtertrace(M, \bar{\tau} ,\filtertrace(M, \bar{\tau}, \bar{\mathcal{D}})) = \filtertrace(M, \bar{\tau}, \bar{\mathcal{D}}).
%     \end{equation*}
% \end{lemma}
% \textit{Proof} This follows directly from Lemma \ref{lem:double-filter}.

\begin{lemma}
    \label{lem:eri-predicate-satisfaction1}
    Let $\mathcal{D}$ be a time point of some trace, where $\mathcal{D}$ is a structure over the signature $S = (C,R,\iota)$ and let $r \in R$ be a predicate. 
    Then 
    \begin{equation*}
        v(\bar{t}) \in r^{\mathcal{D}}
        \text{ iff }
        v(\bar{t}) \in r^{\filter(\ERI(r(\bar{t})), 0, \mathcal{D})},
    \end{equation*}
    where $\bar{t} = (t_1, \dots, t_{\iota(r)})$, and $v(\bar{t}) = (v(t_1), \dots, v(t_{\iota(r)}))$.
\end{lemma}
\textit{Proof} 
This Lemma follows fairly directly from Definition \ref{def:filter} and Definition \ref{def:e-rel-int}.
The extended relative interval of $r(\bar{t})$, $\ERI(r(\bar{t})) = \{r(\mask(\bar{t})) \to [0,0]\}$.
Thus there is a predicate and a mask in $\ERI(r(\bar{t}))$ that do match $r(\bar{t})$ with the interval $[0,0]$ of which $0$ is trivially an element of.

% \begin{lemma}
%     \label{lem:eri-predicate-satisfaction2}
%     Let $(\bar{\mathcal{D}}, \bar{\tau})$ be a temporal structure over the signature $S = (C,R,\iota)$, and $r \in R$ a predicate.
%     Then $((\bar{\mathcal{D}}, \bar{\tau}), v, i) \models r$ iff $(\filtertrace(\ERI(r), \bar{\tau}, \bar{\mathcal{D}}), v, i) \models r$ for any valuation $v$ and $i \in \mathbb{N}$.

% \end{lemma}
% \textit{Proof} \dots

\begin{definition}
    \label{def:mt-slice}
    Let $T \subseteq \mathbb{Z}$ be an interval and $M$ be a masked predicate map where for all $(m, J) \in M$, $J \subseteq T$.
    And let $(\bar{\mathcal{D}}, \bar{\tau})$ a trace.
    The MT-slice of $(\bar{\mathcal{D}}, \bar{\tau})$ is the time slice $(\bar{\mathcal{D}}', \bar{\tau}')$ of $(\bar{\mathcal{D}}, \bar{\tau})$, where $s:[0,l) \to \mathbb{N}$ is the function $s(i') = i' + c, l = |\{i \in \mathbb{N} \mid \tau_i \in T\}|$, and $c = \min\{i \in \mathbb{N} \mid \tau_i \in T \}$. 
    We also require that $\tau_l' \not\in T$ and $\mathcal{D}'_{i'} = \filter(M, \tau_{s(i')}, \mathcal{D}_{s(i')})$, for all $i' \in [0,l)$.
\end{definition}


\begin{definition}
    \label{def:overlapping-ext}
    Let $T \subseteq \mathbb{Z}$ be an interval and $M$ be a masked predicate map where for all $(m, J) \in M$, $J \subseteq T$.
    Let $c,i \in \mathbb{N}$.
    The temporal structures $(\bar{\mathcal{D}}, \bar{\tau})$ and $(\bar{\mathcal{D}}', \bar{\tau}')$ are $(M,T,c,i)$-overlapping if the following conditions hold:
    \renewcommand{\labelenumi}{\arabic{enumi}.}
    \begin{enumerate}
        \item $j \geq c$, $\filter(M, \tau_j, \mathcal{D}_j)=\filter(M, \tau'_{j-c}, \mathcal{D}'_{j-c})$ and $\tau_j = \tau'_{j-c}$, for all $j \in \mathbb{N}$ with $\tau_j - \tau_i \in T$,
        \item $\filter(M, \tau_{j' + c}, \mathcal{D}_{j' + c})=\filter(M, \tau'_{j'}, \mathcal{D}'_{j'})$ and $\tau_{j' + c} = \tau'_{j'}$, for all $j' \in \mathbb{N}$ with $\tau'_{j'} - \tau_i \in T$,
    \end{enumerate}
\end{definition}
Like with Definition \ref{def:overlapping}, two traces are $(M,T,c,i)$-overlapping if their \textit{filtered} time points are the same on an interval of time stamps.
The next lemma establishes that a trace and an $MT$-slice of that trace are $(M,T,c,i)$-overlapping.
It is analogous to Lemma A.2 in Basin et al. \cite{Basin2016}.
\begin{lemma}
    \label{lem:mt-slice-overlap}
    Let $T \subseteq \mathbb{Z}$ be an interval and let $M$ be a masked predicate map where for all $(m,I) \in M$, $I \subseteq T$.
    Let $(\bar{\mathcal{D}}, \bar{\tau})$ a trace, and $(\bar{\mathcal{D}}', \bar{\tau}')$ a $(M, T)$-slice of $(\bar{\mathcal{D}}, \bar{\tau})$.
    The traces $(\bar{\mathcal{D}}, \bar{\tau})$ and $(\bar{\mathcal{D}}', \bar{\tau}')$ are $(M,T,c,i)$-overlapping, for all $i \in \mathbb{N}$ with $\tau_i \in T$, where $c \in \mathbb{N}$ is the value in Definition \ref{def:mt-slice} used by the function $s$.
\end{lemma}
\textit{Proof}
% TODO change proof to correspond to updated lemma
The precondition in Definition \ref{def:overlapping-ext}, that for all intervals in $M$ are a subset of the larger interval $T$ is satisfied, as it is also a precondition for Lemma \ref{lem:mt-slice-overlap}.
First we show condition 1 in Definition \ref{def:overlapping-ext}.
\dots


Analogous to Lemma A.3 in Basin et al. \cite{Basin2016} the following Lemma states that any overlapping traces still overlap, also overlap on parts of of the overlapping section.
\begin{lemma}
    \label{lem:eri-overlap-transitivity}
    Let $(\bar{\mathcal{D}}, \bar{\tau})$ and $(\bar{\mathcal{D}}', \bar{\tau}')$ be two traces that are $(M,T,c,i)$-overlapping, for some masked predicate map $M$, some $T \subseteq \mathbb{Z}$, $c \in \mathbb{N}$, and $i \in \mathbb{Z}$.
    Then $(\bar{\mathcal{D}}, \bar{\tau})$ and $(\bar{\mathcal{D}}', \bar{\tau}')$ are $(N, K, c, k)$-overlapping, for each $k \in \mathbb{N}$ with $\tau_k - \tau_i \in T$, $K \subseteq \{ \tau_i - \tau_k \} \oplus T$, and $N \subseteqmap \{ \tau_i - \tau_k \} \oplusext M$.
    % TODO check that this lemma is complete
\end{lemma}
\textit{Proof} 
% TODO write this proof
\dots

With this we get to the key Lemma that we rely on to make our optimization when selecting and querying a subtrace from the database.
It is analogous to Lemma A.4 in Basin et al. \cite{Basin2016}.

\begin{lemma}
    \label{lem:eri-overlap}
    Let $\phi$ be a formula and $(\bar{\mathcal{D}}, \bar{\tau})$ and $(\bar{\mathcal{D}}', \bar{\tau}')$ temporal structures.
    If $(\bar{\mathcal{D}}, \bar{\tau})$ and $(\bar{\mathcal{D}}', \bar{\tau}')$ are $(\ERI(\phi), \RI(\phi), c, i)$-overlapping, for some $c$ and $i$, then for all valuations $v$, it holds that $(\bar{\mathcal{D}},\bar{\tau},v,i) \models \phi$ iff $(\bar{\mathcal{D}}', \bar{\tau}', v, i-c) \models \phi$.
\end{lemma}
This Lemma applies to MFOTL formulas.
It can be extended to MFODL as well, but it needs a mutually recursive extensions to include regular expressions.
We focus on the core MFOTL formulas.

\textit{Proof} This proof is in large parts analogous to the one to Lemma A.4 in Basin et al. \cite{Basin2016}. First we note that for the same reason as in Lemma A.4 in Basin et al. \cite{Basin2016}, the lemma's statement is well-defined, as the definition of $\RI(\phi)$ did not change and still includes $0$.
We proof the lemma by structural induction on the formula $\phi$.
In this proof $S = (C, R, \iota)$ is a signature, with constants $C$, predicats $R$, and the arity function $\iota$.
Similarly $V$ is the set of variables.
We have the following cases.

\begin{itemize}
    \item 
        $t \approx t'$, where $t, t' \in V \cup C$. 
        The satisfaction of $t \approx t'$ depends only on the valuation $v$.
        Therefore it follows that $(\bar{\mathcal{D}},\bar{\tau},v,i) \models t \approx t'$ iff $(\bar{\mathcal{D}}',\bar{\tau}',v,i-c) \models t \approx t'$.
    \item 
        $t \prec t'$ and $t \preceq$ are analogous to the first one and their detailed proofs are ommitted.
    \item 
        $r{\bar(t)}$, where $t_1, \dots, t_{\iota(r)} \in V \cup C$. 
        Since $(\bar{\mathcal{D}},\bar{\tau})$ and $(\bar{\mathcal{D}}',\bar{\tau}')$ are $(\ERI(\phi), \RI(\phi), c, i)$-overlapping it follows from Condition 1 in Definition \ref{def:overlapping-ext} that 
        \begin{equation*}
            \filter(\ERI(\phi), \tau_i, \mathcal{D}_i) = \filter(\ERI(\phi), \tau'_{i-c} \mathcal{D}'_{i-c})
        \end{equation*}
        and
        \begin{equation*}
            \tau_i = \tau'_{i-c}.
        \end{equation*}
        From Lemma \ref{lem:eri-predicate-satisfaction1} it follows that 
        \begin{equation*}
            (v(t_1), \dots, v(t_{\iota(r)})) \in r^{\mathcal{D}_i}
            \text{ iff }
            (v(t_1), \dots, v(t_{\iota(r)})) \in r^{\filter(\ERI(r(\bar{t})), \tau_i, \mathcal{D}_i)}.
        \end{equation*}
        And similarly
        \begin{equation*}
            (v(t_1), \dots, v(t_{\iota(r)})) \in r^{\mathcal{D}'_{i-c}}
            \text{ iff }
            (v(t_1), \dots, v(t_{\iota(r)})) \in r^{\filter(\ERI(r(\bar{t})), \tau'_{i-c}, \mathcal{D}'_{i-c})}.
        \end{equation*}
        From this and Definiton \ref{def:mfotl-evaluation} it follows that 
        \begin{align*}
            (\bar{\mathcal{D}}, \bar{\tau}, v, i) \models r(\bar{t})
            &\text{ iff } 
            (v(t_1), \dots, v(t_{\iota(r)})) \in r^{\mathcal{D}_i} \\
            &\text{ iff }
            (v(t_1), \dots, v(t_{\iota(r)})) \in r^{\filter(\ERI(r(\bar{t})), \tau_i, \mathcal{D}_i)} = r^{\filter(\ERI(r(\bar{t})), \tau'_{i-c}, \mathcal{D}'_{i-c})}.
        \end{align*}
        It further follows that 
        \begin{align*}
            (\bar{\mathcal{D}}', \bar{\tau}', v, i-c) \models r
            &\text{ iff }
            (v(t_1), \dots, v(t_{\iota(r)})) \in r^{\mathcal{D}'_{i-c}} \\
            &\text{ iff }
            (v(t_1), \dots, v(t_{\iota(r)})) \in r^{\filter(\ERI(r), \tau'_{i-c}, \mathcal{D}'_{i-c})} = r^{\filter(\ERI(r(\bar{t})), \tau_i, \mathcal{D}_i)}.
        \end{align*}
        Thus
        \begin{align*}
            (\bar{\mathcal{D}}, \bar{\tau}, v, i) \models r(\bar{t})
            &\text{ iff }
            (\bar{\mathcal{D}}', \bar{\tau}', v, i-c) \models r(\bar{t})
        \end{align*}
        holds for all valuations $v$ and $i \in \mathbb{N}$.

    \item 
        $\neg \psi$. 
        $(\bar{\mathcal{D}}, \bar{\tau})$ and $(\bar{\mathcal{D}}', \bar{\tau}')$ are $(\ERI(\phi), \RI(\phi), c, i)$-overlapping.
        By definition $\ERI(\neg \psi) = \ERI(\psi)$ and $\RI(\neg \psi) = \RI(\psi)$.
        By the inductive hypothesis $(\bar{\mathcal{D}}, \bar{\tau}, v, i) \models \psi$ iff $(\bar{\mathcal{D}}', \bar{\tau}', v, i-c) \models \psi$ for all valuations $v$.
        Therfore $(\bar{\mathcal{D}}, \bar{\tau}, v, i) \models \neg \psi$ iff $(\bar{\mathcal{D}}', \bar{\tau}', v, i-c) \models \neg \psi$ for all valuations $v$.

    \item 
        $\psi \lor \chi$.
        $(\bar{\mathcal{D}}, \bar{\tau})$ and $(\bar{\mathcal{D}}', \bar{\tau}')$ are $(\ERI(\psi) \Cupmerge \ERI(\chi), \RI(\psi) \Cup \RI(\chi), c, i)$-overlapping.
        % TODO define subseteqmap
        % TODO define analogous lemma to Lemma A.3
        From $\ERI(\psi) \subseteqmap \ERI(\psi) \Cupmerge \ERI(\chi)$, $\ERI(\chi) \subseteqmap \ERI(\psi) \Cupmerge \ERI(\chi)$, $\RI(\psi) \subseteq \RI(\psi) \Cup \RI(\chi)$, and $\RI(\chi) \subseteq \RI(\psi) \Cup \RI(\chi)$,
            it follows from Lemma \ref{lem:eri-overlap-transitivity} that $(\bar{\mathcal{D}}, \bar{\tau})$ and $(\bar{\mathcal{D}}', \bar{\tau}')$ are $(\ERI(\psi), \RI(\psi), c, i)$- and $(\ERI(\chi), \RI(\chi), c, i)$-overlapping.
        By the inductive hypothesis, $(\bar{\mathcal{D}}, \bar{\tau}, v, i) \models \psi$ iff $(\bar{\mathcal{D}}', \bar{\tau}', v, i-c) \models \psi$ and $(\bar{\mathcal{D}}, \bar{\tau}, v, i) \models \chi$ iff $(\bar{\mathcal{D}}', \bar{\tau}', v, i-c) \models \chi$.
        Therefore $(\bar{\mathcal{D}}, \bar{\tau}, v, i) \models \psi \lor \chi$ iff $(\bar{\mathcal{D}}', \bar{\tau}', v, i-c) \models \psi \lor \chi$.
        
    \item 
        $\exists x. \psi$.
        From $\ERI(\exists x. \psi) = \ERI(\psi)$ and $\RI(\exists x. \psi) = \RI(\psi)$ it follows that $(\bar{\mathcal{D}}, \bar{\tau})$ and $(\bar{\mathcal{D}}', \bar{\tau}')$ are $(\ERI(\psi), \RI(\psi), c, i)$-overlapping.
        By the inductive hypothesis we have $(\bar{\mathcal{D}}, \bar{\tau}, v, i) \models \psi$ iff $(\bar{\mathcal{D}}', \bar{\tau}', v, i-c) \models \psi$ for all valuations $v$.
        Thus for all $d$, $(\bar{\mathcal{D}}, \bar{\tau}, v[x \mapsto d], i) \models \psi$ iff $(\bar{\mathcal{D}}', \bar{\tau}', v[x \mapsto d], i-c) \models \psi$.
        By Definition \ref{def:mfotl-evaluation} it follows directly that $(\bar{\mathcal{D}}, \bar{\tau}, v, i) \models \exists x. \psi$ iff $(\bar{\mathcal{D}}', \bar{\tau}', v, i-c) \models \exists x. \psi$ for all valuations $v$.

    \item
        $\Previous{[a,b)} \psi$.
        $(\bar{\mathcal{D}}, \bar{\tau})$ and $(\bar{\mathcal{D}}', \bar{\tau}')$ are $(\ERI(\Previous{[a,b)} \psi), \RI(\Previous{[a,b)} \psi), c, i)$- overlapping, where
        \begin{equation*}
            \ERI(\Previous{[a,b)} \psi) = (-b,0] \Cupext ((-b,-a] \oplusext \ERI(\psi))
        \end{equation*}
        and
        \begin{equation*}
            \RI(\Previous{[a,b)} \psi) = (-b,0] \Cup ((-b,-a] \oplus \RI(\psi))
        \end{equation*}
        From $0 \in \RI(\Previous{[a,b)} \psi)$ and Condition 1 in Definition \ref{def:overlapping-ext}, it follows that $\tau_i = \tau'_{i-c}$.

        We make a case split on the value of $i$. If $i=0$, then $(\bar{\mathcal{D}}, \bar{\tau}, v, i) \not\models \Previous{[a,b)} \psi$, for all valuations $v$.
        From $c \in \mathbb{N}, i-c \in \mathbb{N}$, and $i=0$, it follows that $i-c=0$.
        Since $i-c = 0 = i$ it is trivial that $(\bar{\mathcal{D}}', \bar{\tau}', v, i-c) \not\models \Previous{[a,b)} \psi$ for all valuations $v$.

        Next we consider $i > 0$ and make another case split on whether $\tau_i - \tau_{i-1} \in [a,b)$.
        \begin{itemize}
            \item 
                $\tau_i - \tau_{i-1} \in [a,b)$.
                Then $\tau_{i-1} - \tau_i \in \RI(\Previous{[a,b)} \psi)$ and from Condition 1 in Definition \ref{def:overlapping-ext} it follows that $i-1 \geq c$, $\tau_{i-1} = \tau'_{i-c-1}$, and hence $\tau'_{i-c} - \tau'_{i-c-1} \in [a,b)$.
                It further follows that
                \begin{align*}
                    \ERI(\psi) 
                    &\subseteqmap \{\tau_i - \tau_{i-1} \} \oplusext \{ \tau_{i-1} - \tau_i \} \oplusext \ERI(\psi) \\
                    &\subseteqmap \{\tau_i - \tau_{i-1} \} \oplusext (-b, -a] \oplusext \ERI(\psi) \\
                    &\subseteqmap \{\tau_i - \tau_{i-1} \} \oplusext \ERI(\Previous{[a,b)} \psi)
                \end{align*}
                and
                \begin{align*}
                    \RI(\psi) 
                    &\subseteq \{\tau_i - \tau_{i-1} \} \oplus \{ \tau_{i-1} - \tau_i \} \oplus \RI(\psi) \\
                    &\subseteq \{\tau_i - \tau_{i-1} \} \oplus (-b, -a] \oplus \RI(\psi) \\
                    &\subseteq \{\tau_i - \tau_{i-1} \} \oplus \RI(\Previous{[a,b)} \psi).
                \end{align*}
                Thus by Lemma \ref{lem:eri-overlap-transitivity} $(\bar{\mathcal{D}}, \bar{\tau})$ and $(\bar{\mathcal{D}}', \bar{\tau}')$ are $(\ERI(\psi), \RI(\psi), c, i-1)$- overlapping.
                By the induction hypothesis $(\bar{\mathcal{D}}, \bar{\tau}, v, i-1) \models \psi$ iff $(\bar{\mathcal{D}}', \bar{\tau}', v, i-c-1) \models \psi$ for all valuations $v$.
                Because $\tau_i = \tau'_{i-c}$ and $\tau_{i-1} = \tau'_{i-c-1}$, it follows that $(\bar{\mathcal{D}}, \bar{\tau}, v, i) \models \Previous{[a,b)} \psi$ iff $(\bar{\mathcal{D}}', \bar{\tau}', v, i-c) \models \Previous{[a,b)} \psi$ for all valuations $v$.
            \item 
                $\tau_i - \tau_{i-1} \not\in [a,b)$.
                Then $(\bar{\mathcal{D}}, \bar{\tau}, v, i) \not\models \Previous{[a,b)} \psi$, for all valuations $v$.
                From Condition 1 of Definition \ref{def:overlapping-ext} we still have $i \geq c$.
                We make a case distinction on whether $i=c$ or $i>c$.
                \begin{itemize}
                    \item
                        $i=c$.
                        In this case $i-c = 0$.
                        Therefore $(\bar{\mathcal{D}}, \bar{\tau}, v, i-c) \not\models \Previous{[a,b)} \psi$, for all valuations $v$.
                    \item
                        $i>c$.
                        We proof this case by contradiction.
                        Assume that $\tau'_{i-c} - \tau'_{i-c-1} = \tau'_{i-c} - \tau'_{i-c-1} \in [a,b)$.
                        From Condition 2 in Definition \ref{def:overlapping-ext} it follows that $\tau_{i-1} = \tau'_{i-c-1}$ and hence $\tau_i - \tau_{i-1} = \tau'_{i-c} - \tau'_{i-c-1} \in [a,b)$.
                        This contradicts $\tau_i - \tau_{i-1} \not\in [a,b)$, so it must be the case that $\tau'_{i-c} - \tau'_{i-c-1} \not\in [a,b)$.
                        It follows that $(\bar{\mathcal{D}}', \bar{\tau}', v, i-c) \not\models \Previous{[a,b)} \psi$, for all valuations $v$.
                \end{itemize}
        \end{itemize}
    \item
        $\Next{[a,b)}$.
        The "next" case is in large parts similar to the one for "previous".
        Like with Lemma A.4 \cite{Basin2016} it is simpler than "previous", because no special distinctions have to be made for $i=0$ and $i-c=0$.
    \item
        $\psi \Since{[a,b)} \chi$.
        $(\bar{\mathcal{D}},\bar{\tau})$ and $(\bar{\mathcal{D}}',\bar{\tau}')$ are $(\ERI(\psi \Since{[a,b)} \chi), \RI(\psi \Since{[a,b)} \chi), v, i)$- overlapping, where
        \begin{align*}
            \ERI(\psi \Since{[a,b)} \chi)
            &= (-b,0] \Cupext ((-b,0] \oplusext \ERI(\psi)) \Cupmerge ((-b,-a] \oplusext \ERI(\chi))
        \end{align*}
        and
        \begin{align*}
            \RI(\psi \Since{[a,b)} \chi)
            &= (-b,0] \Cup ((-b,0] \oplus \RI(\psi)) \Cup ((-b,-a] \oplus \RI(\chi)).
        \end{align*}
        From $0 \in \RI(\psi \Since{[a,b)} \chi)$ and Condition 1 in Definition \ref{def:overlapping-ext} it follows that $\tau_i = \tau'_i-c$.

        First we establish the two following claims.
        \begin{enumerate}
            \item
                For all $j \in \mathbb{N}$ with $j \leq i$ and $\tau_i - \tau_j \in [a,b)$, it holds that 
                \begin{align*}
                    \RI(\chi) 
                    &\subseteq \{ \tau_i - \tau_j \} \oplus \{ \tau_j - \tau_i\} \oplus \RI(\chi) \\
                    &\subseteq \{\tau_i - \tau_j \} \oplus (-b, -a] \oplus \RI(\chi) \\
                    &\subseteq \{ \tau_i - \tau_j \} \oplus \RI(\psi \Since{[a,b)} \chi)
                \end{align*}
                as well as
                \begin{align*}
                    \ERI(\chi) 
                    &\subseteqmap \{ \tau_i - \tau_j \} \oplusext \{ \tau_j - \tau_i\} \oplusext \ERI(\chi) \\
                    &\subseteqmap \{\tau_i - \tau_j \} \oplusext (-b, -a] \oplusext \ERI(\chi) \\
                    &\subseteqmap \{ \tau_i - \tau_j \} \oplusext \ERI(\psi \Since{[a,b)} \chi)
                \end{align*}
                and $j \geq c$.
                By Lemma \ref{lem:eri-overlap-transitivity} $(\bar{\mathcal{D}}, \bar{\tau})$ and $(\bar{\mathcal{D}}', \bar{\tau}')$ are $(\ERI(\chi), \RI(\chi), c, j)$- overlapping.
                It then follows from the inductive hypothesis that $(\bar{\mathcal{D}}, \bar{\tau}, v, j) \models \chi$ iff $(\bar{\mathcal{D}}', \bar{\tau}', v, j-c) \models \chi$, for all valuations $v$.
            \item
                For all $k \in \mathbb{N}$ with $k \leq i$ and $\tau_i - \tau_k \in [0,b)$, it holds that
                \begin{align*}
                    \RI(\psi)
                    &\subseteq \{ \tau_i - \tau_k \} \oplus \{ \tau_k - \tau_i \} \oplus \RI(\psi) \\
                    &\subseteq \{ \tau_i - \tau_k \} \oplus (-b,0] \oplus \RI(\psi) \\
                    &\subseteq \{ \tau_i - \tau_k \} \oplus \RI(\psi \Since{[a,b)} \chi) \\
                \end{align*}
                as well as
                \begin{align*}
                    \ERI(\psi)
                    &\subseteqmap \{ \tau_i - \tau_k \} \oplusext \{ \tau_k - \tau_i \} \oplusext \ERI(\psi) \\
                    &\subseteqmap \{ \tau_i - \tau_k \} \oplusext (-b,0] \oplusext \ERI(\psi) \\
                    &\subseteqmap \{ \tau_i - \tau_k \} \oplusext \ERI(\psi \Since{[a,b)} \chi) \\
                \end{align*}
                and $k \geq c$
                By Lemma \ref{lem:eri-overlap-transitivity} $(\bar{\mathcal{D}}, \bar{\tau})$ and $(\bar{\mathcal{D}}', \bar{\tau}')$ are $(\ERI(\psi), \RI(\psi), c, k)$- overlapping.
                It follows from the inductive hypothesis that $(\bar{\mathcal{D}}, \bar{\tau}, v, k) \models \psi$ iff $(\bar{\mathcal{D}}', \bar{\tau}', v, k-c) \models \psi$ for all valuations $v$.
        \end{enumerate}


    \item
        $\psi \Until{[a,b)} \chi$.
\end{itemize}


%%%%%%%%%%%%%%%%%%%%%%%%%%%%%%%%%%%%%%%%%%%%%%%%%%%%%%%%%%%%%%%%%%%%%%%%%

% \begin{lemma}
%     \label{lem:ri-completness}
%     Let $\phi$ be a formula and $(\bar{\mathcal{D}}, \bar{\tau}), (\bar{\mathcal{D}}', \bar{\tau}')$ temporal structures.
%     If $(\bar{\mathcal{D}}, \bar{\tau})$ and $(\bar{\mathcal{D}}', \bar{\tau}')$ are $(\RI(\phi),c,i)$-overlapping, for some $c$ and $i$, then for all valuations $v$, it holds that $(\bar{\mathcal{D}}, \bar{\tau},v,i) \models \phi$ iff $(\bar{\mathcal{D}}',\bar{\tau}',v,i) \models \phi$.
% \end{lemma}

% A proof by structural induction on the structure of MFOTL formulas for Lemma A.4 \cite{Basin2016} / Lemma \ref{lem:ri-completness} is provided in Basin et al. \cite{Basin2016}.
% We will extend this proof to also include the MFODL formulas as we defined them in Defintion \ref{def:mfodl-formula}.

% For this we need an analogous lemma that works for regular expressions.

% \begin{lemma}
%     \label{lem:ri-completness-reg}
%     Let $\rho$ be a regular expression and $(\bar{\mathcal{D}}, \bar{\tau}), (\bar{\mathcal{D}}', \bar{\tau}')$ temporal structures.
%     If $(\bar{\mathcal{D}}, \bar{\tau})$ and $(\bar{\mathcal{D}}', \bar{\tau}')$ are $(\RIr(\rho),c,i)$-overlapping, for some $c$ and $i$, then for all valuations $v$ and $j \in \mathbb{N}$, it holds that $(\bar{\mathcal{D}}, \bar{\tau},v,i,j) \modelsreg \rho$ iff $(\bar{\mathcal{D}}',\bar{\tau}',v,i-c,j-c) \modelsreg \rho$.
% \end{lemma}

%%%%%%%%%%%%%%%%%%%%%%%%%%%%%%%%%%%%%%%%%%%%%%%%%%%%%%%%%%%%%%%%%%%%%%%%%
%%% regular interval lemmata
% We also need Definition A.4 from Basin et al. \cite{Basin2016}.
% \begin{definition}
%     \label{def:overlapping}
%     Let $I \subseteq \mathbb{Z}$ be an interval and $c,i \in \mathbb{N}$.
%     The temporal structures $(\bar{\mathcal{D}}, \bar{\tau})$ and $(\bar{\mathcal{D}}', \bar{\tau}')$ are $(I,c,i)$-overlapping if the following conditions hold:
%     \renewcommand{\labelenumi}{\arabic{enumi}.}
%     \begin{enumerate}
%         \item $j \geq c, \mathcal{D}_j = \mathcal{D}'_{j-c},$ and $\tau_j = \tau'_{j-c},$ for all $j \in \mathbb{N}$ with $\tau_j - \tau_i \in I$.
%         \item $\mathcal{D}_{j'+c} = \mathcal{D}'_j,$ and $\tau_{j'+c} = \tau'_{j'},$ for all $j' \in \mathbb{N}$ with $\tau'_{j'}-\tau_i \in I$.
%     \end{enumerate}
% \end{definition}
% And finally Lemma A.4 itself.
% \begin{lemma}
%     \label{lem:ri-completness}
%     Let $\phi$ be a formula and $(\bar{\mathcal{D}}, \bar{\tau}), (\bar{\mathcal{D}}', \bar{\tau}')$ temporal structures.
%     If $(\bar{\mathcal{D}}, \bar{\tau})$ and $(\bar{\mathcal{D}}', \bar{\tau}')$ are $(\RI(\phi),c,i)$-overlapping, for some $c$ and $i$, then for all valuations $v$, it holds that $(\bar{\mathcal{D}}, \bar{\tau},v,i) \models \phi$ iff $(\bar{\mathcal{D}}',\bar{\tau}',v,i) \models \phi$.
% \end{lemma}

%%%%%%%%%%%%%%%%%%%%%%%%%%%%%%%%%%%%%%%%%%%%%%%%%%%%%%%%%%%%%%%%%%%%%%%%%
%%% Justification/ intuition for also getting all time points inside the entire regular interval
% To get a correct sub trace $\sigma'$ we cannot simply extract the sub trace per predicate and mask.
% There are possibly time points and time stamps in the original trace that have no predicate attached to them.
% Therefore we additionally need to extract all such time stamps that fall into the regular relative interval of the formula.
% Empty time points or time stamps can influence the truth value of a formula and can therefore not be omitted.

% The first case of atomic formulas that are not a predicate is trivial.
% They do not depend on any events that may or may not be present in a log and can be evaluated as is.
% Next a simple predicate only depends on the current time point with time stamp $0$.
% For the unary and binary first order logic formulas (negation, quantification, and, or) the extended relative interval is the special union of relative intervals of the sub formula(s).
% Special union referring to the operator that keeps the intervals as intervals.