\section{Relative Intervals}

First we append the definition of relative intervals from Basin et al. \cite{Basin2016} to include all operators currently supported by MonPoly.
Namely we add definitions for the MFODL operators.
Intervals are defined over $\mathbb{Z}$ and can either be open or closed.
The operators $\oplus$ and $\Cup$ are defined the same way as in Basin et al. \cite{Basin2016}.
Let $I$ and $J$ be some arbitrary intervals then $I \oplus J := \{i+j \mid i \in I \text{ and } j \in J\}$ and $I \Cup J$ is the smallest interval containing all values in both $I$ and $J$.

\begin{definition} 
    \label{def:rel-int}
    The relative interval of the formula $\phi$, $\RI(\phi) \subseteq \mathbb{Z}$ is defined recursively over the formula structure: 
    \begin{equation*}
        \RI(\phi) =
        \begin{cases}
            \{0\}     & \text{if $\phi$ is an atomic formula,} \\ 
            \RI(\psi) & \text{if $\phi$ is of the form $\neg \psi$, 
                                $\exists x.\psi$,} \\ &\text{or $\forall x.\psi$,} \\
            \RI(\psi) \Cup \RI(\chi) & \text{if $\phi$ is of the form $\psi \lor \chi, or
                                            \psi \land \chi$,} \\
            (-b,0] \Cup ((-b,-a] \oplus \RI(\psi)) & \text{if $\phi$ is of the form $\Previous{[a,b)}\psi$,} \\
            [0,b) \Cup ([a,b) \oplus \RI(\psi)) & \text{if $\phi$ is of the form $\Next{[a,b)}$,}\\
            (-b,0] \Cup ((-b,0] \oplus \RI(\psi)) \Cup ((-b,-a] \oplus \RI(\chi)) & \text{if $\phi$ is of the form $\psi \Since{}_{[a,b)} \chi$,} \\
            [0,b) \Cup ([0,b) \oplus \RI(\psi)) \Cup ([a,b) \oplus \RI(\chi)) & \text{if $\phi$ is of the form $\psi \Until{[a,b)} \chi$,} \\
            [0,b) \Cup ([0,b) \oplus \RIr(\rho)) & \text{if $\phi$ is of the form $\Fregex{[a,b)} \rho$, and}\\
            (-b,0] \Cup ((-b,0] \oplus \RIr(\rho)) & \text{if $\phi$ is of the form $\Pregex{[a,b)} \rho$.}\\
        \end{cases}
    \end{equation*}
\end{definition}

We recursively define the relative interval of regular expressions as seen in Basin et al. \cite{Basin2020} in the following way.

\begin{definition}
    \label{def:rel-int-reg}
    The relative interval of the regular expression $\rho$, $\RIr(\rho) \subseteq \mathbb{Z}$ is defined recursively over the structure of the regular expression:
    \begin{equation*}
        \RIr(\rho) =
        \begin{cases}
            \{0\} & \text{if $\rho$ is of the form $\star^k$,} \\
            \RI(\phi) & \text{if $\rho$ is of the form $\phi?$,} \\
            \RIr(\sigma) \Cup \RIr(\tau) & \text{if $\rho$ is of the form $\sigma + \tau$ or $\sigma \cdot \tau$, and} \\
            \RIr(\sigma) & \text{if $\rho$ is of the form $\sigma^*$.}

        \end{cases}
    \end{equation*}
\end{definition}


% TODO
% Basin et al. \cite{Basin2016} provides an intuition for why the definition of $\RI$ makes sense.
We want to show that it is sufficient to extract all time points with a time stamp in the relative interval $\RI(\phi)$ for a formula $\phi$ from a trace to evaluate $\phi$.
Lemma A.4 in Basin et al. \cite{Basin2016} provides exactly this.
The lemma says that the evaluation of a formula $\phi$ is the same on two different temporal structures (traces) for the time point with index $i$ in the first structure and the time point with index $i-c$ in the second structure if they are $(RI(\phi),c,i)$-overlapping as defined by Definition A.4 in Basin et al. \cite{Basin2016}.

For convenience, we restate Lemma A.4 and the definitions it requires here.
First we recall Definition A.2 from Basin et al. \cite{Basin2016}.
\begin{definition}
    \label{def:t-slice}
    Let $T \subseteq \mathbb{Z}$ be an interval and $(\bar{\mathcal{D}}, \bar{\tau})$ a trace.
    The T-slice of $(\bar{\mathcal{D}}, \bar{\tau})$ is the time slice $(\bar{\mathcal{D}'}, \bar{\tau}')$ of $(\bar{\mathcal{D}}, \bar{\tau})$, 
        where $s: [0,l) \to \mathbb{N}$ is the function $s(i') = i' + c, l = |\{i \in \mathbb{N} \mid \tau_i \in T \}|$ and $c = \min\{i \in \mathbb{N} \mid \tau_i \in T \}$.
    We also require that $\tau_l' \not\in T$ and $\mathcal{D}'_{i'} = \mathcal{D}_{s(i')}$, for all $i' \in [0,l)$.
\end{definition}
We also need Definition A.4 from Basin et al. \cite{Basin2016}.
\begin{definition}
    \label{def:overlapping}
    Let $I \subseteq \mathbb{Z}$ be an interval and $c,i \in \mathbb{N}$.
    The temporal structures $(\bar{\mathcal{D}}, \bar{\tau})$ and $(\bar{\mathcal{D}}', \bar{\tau}')$ are $(I,c,i)$-overlapping if the following conditions hold:
    \renewcommand{\labelenumi}{\arabic{enumi}.}
    \begin{enumerate}
        \item $j \geq c, \mathcal{D}_j = \mathcal{D}'_{j-c},$ and $\tau_j = \tau'_{j-c},$ for all $j \in \mathbb{N}$ with $\tau_j - \tau_i \in I$.
        \item $\mathcal{D}_{j'+c} = \mathcal{D}'_j,$ and $\tau_{j'+c} = \tau'_{j'},$ for all $j' \in \mathbb{N}$ with $\tau'_{j'}-\tau_i \in I$.
    \end{enumerate}
\end{definition}

Lemma A.1 in Basin et al. \cite{Basin2016} is useful, we restate it here.
\begin{lemma}
    \label{lem:zero-ri}
    For every formula $\phi$, $0 \in \RI(\phi)$.
\end{lemma}
The proof that this holds is a structural induction over the formula structure for both MFOTL and MFODL, except that the MFODL part requires a similar lemma for regular expressions.

\begin{lemma}
    \label{lem:zero-ri-reg}
    For every regular expression $\rho$, $0 \in \RIr(\rho)$.
\end{lemma}
% The proof of this lemma is a simple structural induction over the structure of regular expressions and is omitted here.
\textit{Proof} the first case is a regular expression $\rho = \star^k$ for $k \in \mathbb{N}$.
Trivially $0 \in \RIr(\rho) = \RIr(\star^k) = {0}$.
Next $\rho = \phi?$ for a formula $\phi$.
$\RIr(\rho) = \RIr(\phi?) = \RI(\phi)$.
By Lemma \ref{lem:zero-ri} $0 \in \RI(\phi)$ and thus $0 \in \RIr(\rho)$.
Now we consider $\rho = \sigma \Cup \tau$ for two regular expressions $\sigma$ and $\tau$.
By the induction hypothesis (IH) $0 \in \RIr(\sigma)$ and $0 \in \RIr(\tau)$.
Thus $0 \in \RIr(\sigma) \Cup \RIr(\tau) = \RIr(\rho)$.
Finally $\rho = \sigma *$ for a regular expression $\sigma$.
By IH $0 \in \RIr(\sigma) = \RIr(\rho)$.
Therefore Lemma \ref{lem:zero-ri-reg} holds for all regular expressions.

And finally Lemma A.4 itself.
\begin{lemma}
    \label{lem:ri-completness}
    Let $\phi$ be a formula and $(\bar{\mathcal{D}}, \bar{\tau}), (\bar{\mathcal{D}}', \bar{\tau}')$ temporal structures.
    If $(\bar{\mathcal{D}}, \bar{\tau})$ and $(\bar{\mathcal{D}}', \bar{\tau}')$ are $(\RI(\phi),c,i)$-overlapping, for some $c$ and $i$, then for all valuations $v$, it holds that $(\bar{\mathcal{D}}, \bar{\tau},v,i) \models \phi$ iff $(\bar{\mathcal{D}}',\bar{\tau}',v,i) \models \phi$.
\end{lemma}

A proof by structural induction on the structure of MFOTL formulas for Lemma A.4 \cite{Basin2016} / Lemma \ref{lem:ri-completness} is provided in Basin et al. \cite{Basin2016}.

For this we need an analogous lemma that works for regular expressions.

\begin{lemma}
    \label{lem:ri-completness-reg}
    Let $\rho$ be a regular expression and $(\bar{\mathcal{D}}, \bar{\tau}), (\bar{\mathcal{D}}', \bar{\tau}')$ temporal structures.
    If $(\bar{\mathcal{D}}, \bar{\tau})$ and $(\bar{\mathcal{D}}', \bar{\tau}')$ are $(\RIr(\rho),c,i)$-overlapping, for some $c$ and $i$, then for all valuations $v$ and $j \in \mathbb{N}$, it holds that $(\bar{\mathcal{D}}, \bar{\tau},v,i,j) \modelsreg \rho$ iff $(\bar{\mathcal{D}}',\bar{\tau}',v,i-c,j-c) \modelsreg \rho$.
\end{lemma}

\textit{Proof}
Lemmas \ref{lem:ri-completness} and \ref{lem:ri-completness-reg} are mutually recursive and must be proofed in combination by structural induction on both the structure of MFODL formulas (Definition \ref{def:mfodl-formula}) and the structure of regular expressions (Definition \ref{def:regex}).
For time reasons we forgo this proof and focus on our extension of the idea behind relative intervals in the next section.
% Note that $(\bar{\mathcal{D}}', \bar{\tau}', v, i-c, j-c) \modelsreg \rho$ is defined.

% We write $\sigma = (\bar{\mathcal{D}}, \bar{\tau})$ and $\sigma' = (\bar{\mathcal{D}}', \bar{\tau}')$ as shorthand for the two temporal structures.
% With the help of Lemma \ref{lem:ri-completness-reg} we now complete the proof for Lemma \ref{lem:ri-completness} with MFODL formulas.
% We continue the proof for the case $\phi = \Pregex{[a,b)} \rho$ where $\rho$ is a regular expression.
% Let $\sigma$ and $\sigma' $ are $ (\RI(\phi),c,i)-$overlapping.
% Where $\RI(\phi) = \RI(\Pregex{[a,b)} \rho) = (-b,0] \Cup ((-b,0] \oplus \RIr(\rho))$.

% \begin{alignat*}{3}
%     (\sigma, v, i) \models \phi \quad
%     &\Longleftrightarrow \quad
%         &&(\sigma, v, i) \models \Pregex{[a,b)} \rho \\
%     &\overset{\text{def \ref{def:mfodl-eval}}}{\Longleftrightarrow} \quad
%         &&j \leq i, \tau_i - \tau_j \in [a,b) \text{ and } (\sigma, v, j, i) \modelsreg \rho \\
%     &\overset{\text{lem \ref{lem:ri-completness-reg}}}{\Longleftrightarrow} \quad
%         &&j \leq i, \tau_i - \tau_j \in [a,b) \text{ and } (\sigma', v, j, i) \modelsreg \rho \\
% \end{alignat*}









% We aim provide a more formal proof of correctness.
% \begin{definition}
%     \label{def:filter}
%     We call filtering of a trace $\sigma=((\tau_i,D_i))_i$ according to an interval $I$ at time point $j$ the trace 
%     \begin{equation*}
%         \filter(\sigma, I, j) = \sigma' = ((\tau_k,D_k), ..., (\tau_j,D_j))
%     \end{equation*}
%     where $k$ is the smallest index for which $\tau_k - \tau_j \in I \cap (-\infty,0]$.
%     % where $k$ is the largest index for which $\tau_j - \tau_k \in I$.
% \end{definition}
% % TODO (N.B. You need your semantics to be defined on finite traces also!)
% The intuition for why we intersect the interval $I$ with all negative numbers, is that this limits the range of possible time points to ones that are in the past from the perspective of $\tau_j$.
% This is sensible, because the set of future events should be empty anyways and we can thus disregard it directly.
% % TODO remove this in document comment
% % \textit{comment: this restriction isn't strictly necessary, it might just over complicate things}.
% \begin{theorem}
%     \label{theo:filter-correctness}
%     For all  $\sigma, v, i, \phi$:
%     \begin{equation*}
%         (\sigma, v, i) \models \phi
%         \quad
%         \Longleftrightarrow 
%         \quad
%         (\filter(\sigma, I, i), v, i) \models \phi
%     \end{equation*}
%     where $I$ is an in interval over $\mathbb{Z}$ with $\RI(\phi) \subseteq I$.
% \end{theorem}
% We prove that Theorem \ref{theo:filter-correctness} holds for any MFOTL or MFODL formula $\phi$ by structural induction over the formula structure of MFOTL/MFODL as defined in the definition \ref{def:mfotl-formula} and Definition \ref{def:mfodl-formula}.
% In the following $\sigma$ is a trace $(\bar{\mathcal{D}}, \bar{\tau})$, $v$ is a valuation mapping, and $i$ is a time point.

% First we consider the case $\phi = t \approx t'$.
% \begin{alignat*}{1}
%     (\sigma, v, i) \models \phi \quad
%     &\Longleftrightarrow \quad
%         (\sigma, v, i) \models t \approx t' \\
%     &\overset{def \ref{def:mfotl-evaluation}}{\Longleftrightarrow} \quad
%         v(t) = v(t') \\
%     &\overset{def \ref{def:mfotl-evaluation}}{\Longleftrightarrow} \quad
%         (\filter(\sigma, \RI(\phi), i), i, v) \models t \approx v' \\
%     &\Longleftrightarrow \quad
%         (\filter(\sigma, \RI(\phi), i), i, v) \models \phi
% \end{alignat*}
% For the other ordering relations the proof can be done analogously.

% Next we consider $\phi = \phi_1 \lor \phi_2$.
% \begin{alignat*}{3}
%     (\sigma, v, i) \models \phi \quad
%     &\Longleftrightarrow \quad
%         &&(\sigma, v, i) \models \phi_1 \lor \phi_2 \\
%     &\overset{def \ref{def:mfotl-evaluation}}{\Longleftrightarrow} \quad
%         &&(\sigma, v, i) \models \phi_1 \text{ or } (\sigma, v, i) \models \phi_2 \\
%     &\overset{\text{IH}}{\Longleftrightarrow} \quad
%         &&(\filter(\sigma, I, i), v, i) \models \phi_1 \text{ or } 
%         (\filter(\sigma, I, i), v, i) \models \phi_2 \\
%         &&&\text{ for } \RI(\phi_1), \RI(\phi_2) \subseteq I\\
%     &\overset{}{\Longleftrightarrow} \quad
%         &&(\filter(\sigma, I, i), v, i) \models \phi_1 \lor \phi_2 
%         \text{ for } \RI(\phi_1), \RI(\phi_2) \subseteq I\\
%     &\overset{def \ref{def:mfotl-evaluation}}{\Longleftrightarrow} \quad
%         &&(\filter(\sigma, I, i), v, i) \models \phi 
%         \text{ for } \RI(\phi_1), \RI(\phi_2) \subseteq I\\
% \end{alignat*}
% To complete the proof we need to show that $\RI(\phi) = \RI(\phi_1 \lor \phi_2) = \RI(\phi_1) \Cup \RI(\phi_2) \subseteq I$.
% This follows almost directly from the condition that $\RI(\phi_1), \RI(\phi_2) \subseteq I$ and the definition of $J_a \Cup J_b$ for two intervals $J_a$ and $J_b$ as the smallest interval that contains all elements in either $J_a$ or $J_b$.
% We already know that all elements in the two relative intervals for $\phi_1$ and $\phi_2$ are all in $I$.
% The only elements potentially unaccounted for are ones in between the two relative intervals, but since $I$ is an interval it cannot have any gaps and thus $(\phi_1) \Cup \RI(\phi_2) \subseteq I$ must be true.

% The "and" case would be done the same way and is omitted here.

% For $\phi = \exists x . \phi'$.
% \begin{alignat*}{3}
%     (\sigma, v, i) \models \phi \quad
%     &\Longleftrightarrow \quad
%         &&(\sigma, v, i) \models \exists x. \phi' \\
%     &\overset{\text{def \ref{def:mfotl-evaluation}}}{\Longleftrightarrow} \quad
%         &&(\sigma, v[x \mapsto d], i) \models \phi' \text{ for some } d \in |\bar{\mathcal{D}}| \\
%     &\overset{\text{IH}}{\Longleftrightarrow} \quad
%         &&(\filter(\sigma, I, i), v[x \mapsto d], i) \models \phi' \text{ for some } d \in |\bar{\mathcal{D}}| \\
%         &&&\text{ for } \RI(\phi') \subseteq I\\
%     &\overset{\text{def \ref{def:mfotl-evaluation}}}{\Longleftrightarrow} \quad
%         &&(\filter(\sigma, I, i), v, i) \models  \exists x . \phi' \\
%         &&&\text{ for } \RI(\phi') \subseteq I\\
%     &\overset{}{\Longleftrightarrow} \quad
%         &&(\filter(\sigma, I, i), v, i) \models \phi \\
%         &&&\text{ for } \RI(\phi') \subseteq I\\
%     &\overset{(*)}{\Longleftrightarrow} \quad
%         &&(\filter(\sigma, I, i), v, i) \models \phi \\
%         &&&\text{ for } \RI(\phi) \subseteq I\\
% \end{alignat*}
% $(*)$ follows from the definition of the relative interval for the exists quantifier as $\RI(\phi) = \RI(\exists x . \phi') = \RI(\phi')$ 
% The proof for the "for all" quantifier goes the same way and is omitted here.

% Now we look at negation, $\phi = \neg \phi'$.
% \begin{alignat*}{3}
%     (\sigma, v, i) \models \phi \quad
%     &\Longleftrightarrow \quad
%         &&(\sigma, v, i) \models \neg \phi' \\
%     &\overset{\text{def \ref{def:mfotl-evaluation}}}{\Longleftrightarrow} \quad
%         &&(\sigma, v, i) \not\models \phi' \\
%     &\overset{\text{IH}}{\Longleftrightarrow} \quad
%         &&(\filter(\sigma, I, i), v, i) \not\models \phi' \text{ for } \RI(\phi') \subseteq I\\
%     &\overset{\text{def \ref{def:mfotl-evaluation}}}{\Longleftrightarrow} \quad
%         &&(\filter(\sigma, I, i), v, i) \models   \neg \phi' \text{ for } \RI(\phi') \subseteq I\\
%     &\overset{}{\Longleftrightarrow} \quad
%         &&(\filter(\sigma, I, i), v, i) \models \phi \text{ for } \RI(\phi') \subseteq I\\
%     &\overset{(*)}{\Longleftrightarrow} \quad
%         &&(\filter(\sigma, I, i), v, i) \models \phi \text{ for } \RI(\phi) \subseteq I\\
% \end{alignat*}
% Like with the exists quantifier the last transition $(*)$ holds, because $\RI(\phi) = \RI(\neg \phi')= \RI(\phi')$.

% The next proofs are a bit more involved as they concern the temporal operators.
% Consider $\phi = \Previous{J} \phi'$.
% \begin{alignat*}{3}
%     (\sigma, v, i) \models \phi \quad
%     &\Longleftrightarrow \quad
%         &&(\sigma, v, i) \models \Previous{J} \phi' \\
%     &\overset{\text{def \ref{def:mfotl-evaluation}}}{\Longleftrightarrow} \quad
%         && i > 0, \tau_i - \tau_{i+1} \in J, 
%         \text{ and } (\sigma, v, i-1) \models \phi'\\
%     &\overset{\text{IH}}{\Longleftrightarrow} \quad
%         && i > 0, \tau_i - \tau_{i+1} \in J, 
%         \text{ and } (\filter(\sigma, I', i-1), v, i-1) \models \phi' \\
%         &&& \text{for } \RI(\phi') \subseteq I' \\
%     &\overset{\text{def \ref{def:mfotl-evaluation}}}{\Longleftrightarrow} \quad
%         && (\filter(\sigma, I', i-1), v, i) \models \Previous{J} \phi' \\
%         &&& \text{for } \RI(\phi') \subseteq I' \\
%     &\overset{}{\Longleftrightarrow} \quad
%         && (\filter(\sigma, I', i-1), v, i) \models \Previous{J} \phi' \\
%         &&& \text{for } \RI(\phi') \subseteq I' \\
% \end{alignat*}





% Our induction hypothesis is that Theorem \ref{theo:filter-correctness} holds for any subformulas.
% We then show that Theorem \ref{theo:filter-correctness} holds for any formula $\phi$.
% Only formulas of the structures $(\Previous{I} \psi),(\Next{I} \psi), (\psi \Since{I} \psi'), (\psi \Until{I} \psi'), (\Pregex{I} \rho), \text{ and } (\Fregex{I} \rho)$ depend on the trace $\sigma$.
% Therefore Theorem \ref{theo:filter-correctness} holds trivially if $\phi$ has any other structure (first order logic and relations).

% We start with $\phi = (\Previous{[a,b)} \psi)$.
% There are two cases to consider.
% Either the left hand side (LHS) is true, i.e. ($\sigma, v, i \models \phi$) or it is not true, i.e. ($\sigma, v, i \not\models \phi$).
% We first consider the LHS to be true, that is $\phi$ evaluates to true at the time point $i$ with the valuation mapping $v$ and the trace $\sigma$.
% The relative interval of $\phi=(\Previous{[a,b)} \psi)$ is defined as $\RI(\phi) = (-b,0] \Cup ((-b,-a] \oplus \RI(\psi))$.
% By our induction hypothesis $ (\sigma, v, i) \models \psi \Leftrightarrow (\filter(\sigma, \RI(\psi), i), v, i) \models \psi $ holds.
% By definition $(\filter(\sigma, \RI(\phi), i), v, i) \models \phi$ holds if and only if $i > 0, \tau_i - \tau_{i+1} \in [a,b)$ and $(\sigma, v, i-1) \models \psi$.
% From the LHS holding we know that $i>0$.
% We also know that for the LHS to be evaluated to true, $(\sigma, v, i-1) \models \psi$ must also hold.
% And by the induction hypothesis we therefore get $(\filter(\sigma, \RI(\psi), i), v, i) \models \psi$.
% This means the final condition we must show is that for the filtered trace $\sigma'$, $\tau_i - \tau_{i+1} \in [a,b)$.
% So in other words we must show that the two time points $i$ and $i+1$ with their respective time stamps $\tau_i$ and $\tau_{i+1}$ are within the relative interval of $\phi$.

% \textit{comment: is the numbering of time points generally without gaps?}

% If we go with the definition of filtering only applying to the past we can essentially disregard the future operators next and future match.




% The definition of relative intervals is correct if any formula $\phi$ evaluated on a trace $\sigma$ has the same truth value as evaluated on a trace $\sigma'$ that is a filtered version of $\sigma$ based on $\RI(\phi)$.
% Let $\RI(\phi) = [a,b)$ where $a,b \in \mathbb{Z}, \, a \leq 0, \, b > 0$.
% Given that no future events can be extracted from a trace as they have not happened yet, the extracted trace $\sigma'$ only contains time points with time stamps that are in the interval $[a,0]$.
% Formally $\sigma' = \{x \mid x \in \sigma \text{ and } x.ts \in [a,0]\}$, where $x.ts$ denotes the time stamp of $x$.

% The future match operator $\Fregex{I} \rho$ depends on the relative interval of $\rho$ shifted by the interval $I$.
% Similarly the past match operator $\Pregex{I} \rho$ depends on the relative interval of $\rho$ shifted by the inverted interval $I$.
% The interpretation of the relative intervals for regular expressions is a bit different from that of a formula.
% Its relevance is the amount that the "parent" interval of a past or future match operator needs to be shifted by.
% The wildcard operator $\star^k$ doesn't have any interval information attached.
% The parameter $k$ refers to a number of timepoints, but not directly to time stamps.
% The test operator $\phi ?$ is more involved.
% It evaluates a formula $\phi$ which of course has a regular relative interval.
% Depending on which match operator is used we want to evaluate $\phi$ with any starting point inside the specified past or future interval.
