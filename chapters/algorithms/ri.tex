\section{Relative Intervals}

First we append the definition of relative intervals from Basin et al. \cite{Basin2016} to include all operators currently supported by MonPoly.
Namely we add definitions for the MFODL operators.
Intervals are defined over $\mathbb{Z}$ and can either be open or closed.
The operators $\oplus$ and $\Cup$ are defined the same way as in Basin et al. \cite{Basin2016}.
Let $I$ and $J$ be some arbitrary intervals then $I \oplus J := \{i+j \mid i \in I \text{ and } j \in J\}$ and $I \Cup J$ is the smallest interval containing all values in both $I$ and $J$.

\begin{definition} 
    The relative interval of the formula $\varphi$, $\RI(\varphi) \subseteq \mathbb{Z}$ is defined recuresively over the formula structure: 
    \begin{equation*}
        \RI(\varphi) =
        \begin{cases}
            \{0\}     & \text{if $\varphi$ is an an atomic formula,} \\ 
            \RI(\psi) & \text{if $\varphi$ is of the form $\neg \psi$, 
                                $\exists x.\psi$,} \\ &\text{or $\forall x.\psi$,} \\
            \RI(\psi) \Cup \RI(\chi) & \text{if $\varphi$ is of the form $\psi \lor \chi, or
                                            \psi \land \chi$,} \\
            (-b,0] \Cup ((-b,-a] \oplus \RI(\psi)) & \text{if $\varphi$ is of the form $\Previous{[a,b)}\psi$,} \\
            [0,b) \Cup ([a,b) \oplus \RI(\psi)) & \text{if $\varphi$ is of the form $\Next{[a,b)}$,}\\
            (-b,0] \Cup ((-b,0] \oplus \RI(\psi)) \Cup ((-b,-a] \oplus \RI(\chi)) & \text{if $\varphi$ is of the form $\psi \Since{}_{[a,b)} \chi$,} \\
            [0,b) \Cup ([0,b) \oplus \RI(\psi)) \Cup ([a,b) \oplus \RI(\chi)) & \text{if $\varphi$ is of the form $\psi \Until{[a,b)} \chi$,} \\
            [0,b) \Cup ([0,b) \oplus \RIr(\rho)) & \text{if $\varphi$ is of the form $\Fregex{[a,b)} \rho$, and}\\
            (-b,0] \Cup ((-b,0] \oplus \RIr(\rho)) & \text{if $\varphi$ is of the form $\Pregex{[a,b)} \rho$.}\\
        \end{cases}
    \end{equation*}
\end{definition}

We recursively define the relative interval of regular expressions as seen in Basin et al. \cite{Basin2020} in the following way.

\begin{definition}
    The relative interval of the regular expression $\rho$, $\RIr(\rho) \subseteq \mathbb{Z}$ is defined recursively over the structure of the regular expression:
    \begin{equation*}
        \RIr(\rho) =
        \begin{cases}
            \{0\} & \text{if $\rho$ is of the form $\star^k$,} \\
            \RI(\varphi) & \text{if $\rho$ is of the form $\varphi?$,} \\
            \RIr(\sigma) \Cup \RIr(\tau) & \text{if $\rho$ is of the form $\sigma + \tau$ or $\sigma \cdot \tau$, and} \\
            \RIr(\sigma) & \text{if $\rho$ is of the form $\sigma^*$.}

        \end{cases}
    \end{equation*}
\end{definition}

\begin{definition}
    We call filtering of a trace $\sigma=((\tau_i,D_i))_i$ according to an interval $[0,a]$ at time point $j$ the trace 
    \begin{equation*}
        \filter(\sigma, [0,a], j) = \sigma' = ((\tau_k,D_k), ..., (\tau_j,D_j))
    \end{equation*}
    where $k$ is the largest index for which $\tau_i - \tau_k < a$.
\end{definition}
(N.B. You need your semantics to be defined on finite traces also!)

\begin{theorem}
    For all  $\sigma, v, i$:
    \begin{equation*}
        \sigma, v, i |= \varphi => filter(\sigma, \RI(\phi), i), v, i |= \varphi
    \end{equation*}
\end{theorem}

\subsection{Correctness}
% TODO
Basin et al. \cite{Basin2016} provides an intuition for why the definition of $\RI$ is correct for the MFOTL operators.
Here we provide the same for the added MFODL \cite{Basin2020} operators and the regular expressions.

The definition of relative intervals is correct if any formula $\varphi$ evaluated on a trace $\sigma$ has the same truth value as evaluated on a trace $\sigma'$ that is a filtered version of $\sigma$ based on $\RI(\varphi)$.
Let $\RI(\varphi) = [a,b)$ where $a,b \in \mathbb{Z}, \, a \leq 0, \, b > 0$.
Given that no future events can be extracted from a trace as they have not happened yet, the extracted trace $\sigma'$ only contains time points with time stamps that are in the interval $[a,0]$.
Formally $\sigma' = \{x \mid x \in \sigma \text{ and } x.ts \in [a,0]\}$, where $x.ts$ denotes the time stamp of $x$.

The future match operator $\Fregex{I} \rho$ depends on the relative interval of $\rho$ shifted by the interval $I$.
Similarly the past match operator $\Pregex{I} \rho$ depends on the relative interval of $\rho$ shifted by the inverted interval $I$.
The interpretation of the relative intervals for regular expressions is a bit different from that of a formula.
Its relevance is the amount that the "parent" interval of a past or future match operator needs to be shifted by.
The wildcard operator $\star^k$ doesn't have any interval information attached.
The parameter $k$ refers to a number of timepoints, but not directly to time stamps.
The test operator $\varphi ?$ is more involved.
It evaluates a formula $\varphi$ which of course has a regular relative interval.
Depending on which match operator is used we want to evaluate $\varphi$ with any starting point inside the specified past or future interval.
