\begin{lemma}
    \label{lem:eri-overlap-transitivity}
    Let $(\bar{\mathcal{D}}, \bar{\tau})$
        and
        $(\bar{\mathcal{D}}', \bar{\tau}')$ be two temporal structures that are $(M,I,c,i)$-overlapping,
            for some masked predicate map $M$,
            $I \subseteq \mathbb{Z}$,
            $c \in \mathbb{N}$,
            and $i \in \mathbb{Z}$.
    Then $(\bar{\mathcal{D}}, \bar{\tau})$ and $(\bar{\mathcal{D}}', \bar{\tau}')$
    are $(N, K, c, k)$-overlapping,
        for each $k \in \mathbb{N}$
        with $\tau_k - \tau_i \in I$, $K \subseteq \{ \tau_i - \tau_k \} \oplus I$,
        and $N \subseteqmap \{ \tau_i - \tau_k \} \oplusext M$.
    % TODO check that this lemma is complete
\end{lemma}
\textit{Proof} 
% TODO write this proof
This proof follows a similar structure as the proof of Lemma A.3 in Basin et al. \cite{Basin2016}.
We first show that Condition 1 in Definition \ref{def:overlapping-ext} is satisfied, i.e.
for all $j \in \mathbb{N}$ with $\tau_j - \tau_k \in K$,
the following three things hold:
$j \geq c$,
$\tau_j = \tau'_{j-c}$,
and
$ \filter(N, \mathcal{D}_j, \tau_j, \tau_k) = \filter(N, \mathcal{D}'_{j-c}, \tau'_{j-c}, \tau_k)$.

For all $j \in \mathbb{N}$ with $\tau_j - \tau_k \in K$,
    it follows that $\tau_j - \tau_k + \tau_k - \tau_i \in \{\tau_k - \tau_i \} \oplus K$,
    it then follows that $\tau_j - \tau_i \in \{\tau_k - \tau_i\} \oplus K$. 
From $K \subseteq \{ \tau_i - \tau_k \} \oplus I$,
    we get $\{\tau_k - \tau_i\} \oplus K \subseteq \{\tau_k - \tau_i \} \oplus \{\tau_i - \tau_k \} \oplus I = I$,
    or in short $\tau_j - \tau_i \in I$.
And as we know that $(\bar{\mathcal{D}}, \bar{\tau})$
    and $(\bar{\mathcal{D}}', \bar{\tau}')$ are $(M,I,c,i)$-overlapping,
    it follows from Condition 1 in Definition \ref{def:overlapping-ext} that
    for all $j \in \mathbb{N}$ with $\tau_j - \tau_k \in K$,
    $j \geq c$, $\tau_j = \tau'_{j-c}$,
    and $\filter(M, \mathcal{D}_j, \tau_j, \tau_i) = \filter(M, \mathcal{D}'_{j'}, \tau'_{j'}, \tau_i)$.
We are done with Condition 1, if we can show that
\begin{equation*}
    \filter(M, \mathcal{D}_j, \tau_j, \tau_i) = \filter(M, \mathcal{D}'_{j'}, \tau'_{j'}, \tau_i),
\end{equation*}
for all $j \in \mathbb{N}$ with $\tau_j - \tau_k \in K$ implies 
\begin{equation*}
    \filter(N, \mathcal{D}_j, \tau_j, \tau_k) 
    = \filter(N, \mathcal{D}'_{j-c}, \tau'_{j-c}, \tau_k),  
\end{equation*}
for all $j \in \mathbb{N}$ with $\tau_j - \tau_k \in K$.
The equality of two filtered structures can be shown that for all predicates $r$,
$r^{\filter(N, \mathcal{D}_j, \tau_j, \tau_k)} = r^{\filter(N, \mathcal{D}'_{j'}, \tau_{j'}, \tau_k)}$.

By Definition \ref{def:filter} an interpretation $x$ is in $r^{\filter(N, \mathcal{D}_j, \tau_j, \tau_k)}$ iff $\exists p,m,J . ((p,m), J) \in N \land p = r \land m \matches x \land \tau_i - \tau_k \in J$.
From $N \subseteqmap \{ \tau_i - \tau_k \} \oplusext M$ it follows that for any $p,m,J$, with $((p,m), J) \in N$ there must exist $p',m',J'$,
    with $((p',m'), J') \in \{\tau_i - \tau_k \} \oplusext M$,
    $p=p'$,
    $m' \nomoreprecise m$,
    and $J \subseteq J'$.
From the definition of $\oplusext$ it is apparent that
$((p',m'), J') \in \{\tau_i - \tau_k \} \oplusext M$
iff
$((p',m'), \{\tau_k - \tau_i\} \oplus J') \in M$.
Hence there is an interpretation in the filtration by $N$, if it is in the filtration with $M$ shifted by $\{\tau_i - \tau_k\}$.
And from the fact that the two temporal structures are $(M,I,c,i)$- overlapping

%TODO
\textit{Needs to be finished}

\dots


