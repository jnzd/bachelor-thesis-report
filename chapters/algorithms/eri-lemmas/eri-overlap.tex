\begin{lemma}
    \label{lem:eri-overlap}
    Let $\phi$ be a formula and $(\bar{\mathcal{D}}, \bar{\tau})$ and $(\bar{\mathcal{D}}', \bar{\tau}')$ temporal structures.
    If $(\bar{\mathcal{D}}, \bar{\tau})$ and $(\bar{\mathcal{D}}', \bar{\tau}')$ are $(\ERI(\phi), \RI(\phi), c, i)$-overlapping, for some $c$ and $i$, then for all valuations $v$, it holds that $(\bar{\mathcal{D}},\bar{\tau},v,i) \models \phi$ iff $(\bar{\mathcal{D}}', \bar{\tau}', v, i-c) \models \phi$.
\end{lemma}
This Lemma applies to MFOTL formulas.
It can be extended to MFODL as well, but it needs a mutually recursive extensions to include regular expressions.
We focus on the core MFOTL formulas.

\textit{Proof} This proof is in large parts analogous to the one to Lemma A.4 in Basin et al. \cite{Basin2016}, and we follow the same structure. First we note that for the same reason as in Lemma A.4 in Basin et al. \cite{Basin2016}, the lemma's statement is well-defined, as the definition of $\RI(\phi)$ did not change and still includes $0$.
We proof the lemma by structural induction on the formula $\phi$.
In this proof $S = (C, R, \iota)$ is a signature, with constants $C$, predicats $R$, and the arity function $\iota$.
Similarly $V$ is the set of variables.
We have the following cases.

\begin{itemize}
    \item 
        $t \approx t'$, where $t, t' \in V \cup C$. 
        The satisfaction of $t \approx t'$ depends only on the valuation $v$.
        Therefore it follows that $(\bar{\mathcal{D}},\bar{\tau},v,i) \models t \approx t'$ iff $(\bar{\mathcal{D}}',\bar{\tau}',v,i-c) \models t \approx t'$.
    \item 
        $t \prec t'$ and $t \preceq$ are analogous to the first one and their detailed proofs are omitted.
    \item 
        $r(\bar{t})$, where $t_1, \dots, t_{\iota(r)} \in V \cup C$. 
        Since $(\bar{\mathcal{D}},\bar{\tau})$ and $(\bar{\mathcal{D}}',\bar{\tau}')$ are $(\ERI(\phi), \RI(\phi), c, i)$-overlapping it follows from Condition 1 in Definition \ref{def:overlapping-ext} that 
        \begin{equation*}
            \filter(\ERI(\phi), \mathcal{D}_j, \tau_j, \tau_i)
            = \filter(\ERI(\phi), \mathcal{D}'_{j-c}, \tau'_{j-c}, \tau_i),
        \end{equation*}
        $\tau_j = \tau'_{j-c}$, and $j \geq c$
        for all $j \in \mathbb{N}$ with $\tau_j - \tau_i \in \RI(\phi)$.
        % We make a case distinction on whether $\tau_j - \tau_i \in \RI(\phi)$.
        % \begin{itemize}
            % \item $\tau_j - \tau_i \in \RI(\phi)$.
                % In this case,
                By Definition \ref{def:filter}, there exists a valuation $v(\bar{t})$ in $r^{\mathcal{D}_j}$ iff there exist $m,J$, with $((r,m),J) \in \ERI(\phi)$ with $\tau_j - \tau_i \in J$, and $m \matches v(\bar{t})$.
                And from the equality of the two filters given to us by Condition 1 of Definition \ref{def:overlapping-ext} this is the case iff there exist $m', J'$, with $((r,m'),J') \in \ERI(\phi)$ with $\tau'_{j-c} - \tau_i \in J'$, and $m' \matches v(\bar{t})$ for all valuations $v$.
                And further this last part holds iff $v(\bar{t}) \in r^{\mathcal{D}'_{j-c}}$.
                Thus, by Definition, \ref{def:mfotl-evaluation}
                    $(\bar{\mathcal{D}},\bar{\tau},v,i) \models r(\bar{t})$
                    iff
                    $(\bar{\mathcal{D}}',\bar{\tau}',v,i-c) \models r(\bar{t})$.
        %     \item $\tau_j - \tau_i \not\in \RI(\phi)$.
        %         In this case it follows from Definition \ref{def:mfotl-evaluation} that
        % \end{itemize}

    \item 
        $\neg \psi$. 
        $(\bar{\mathcal{D}}, \bar{\tau})$ and $(\bar{\mathcal{D}}', \bar{\tau}')$ are $(\ERI(\phi), \RI(\phi), c, i)$-overlapping.
        By definition $\ERI(\neg \psi) = \ERI(\psi)$ and $\RI(\neg \psi) = \RI(\psi)$.
        By the inductive hypothesis $(\bar{\mathcal{D}}, \bar{\tau}, v, i) \models \psi$ iff $(\bar{\mathcal{D}}', \bar{\tau}', v, i-c) \models \psi$ for all valuations $v$.
        Therefore $(\bar{\mathcal{D}}, \bar{\tau}, v, i) \models \neg \psi$ iff $(\bar{\mathcal{D}}', \bar{\tau}', v, i-c) \models \neg \psi$ for all valuations $v$.

    \item 
        $\psi \lor \chi$.
        $(\bar{\mathcal{D}}, \bar{\tau})$ and $(\bar{\mathcal{D}}', \bar{\tau}')$ are $(\ERI(\psi) \Cupmerge \ERI(\chi), \RI(\psi) \Cup \RI(\chi), c, i)$-overlapping.
        From $\ERI(\psi) \subseteqmap \ERI(\psi) \Cupmerge \ERI(\chi)$, $\ERI(\chi) \subseteqmap \ERI(\psi) \Cupmerge \ERI(\chi)$, $\RI(\psi) \subseteq \RI(\psi) \Cup \RI(\chi)$, and $\RI(\chi) \subseteq \RI(\psi) \Cup \RI(\chi)$,
            it follows from Lemma \ref{lem:eri-overlap-transitivity} that $(\bar{\mathcal{D}}, \bar{\tau})$ and $(\bar{\mathcal{D}}', \bar{\tau}')$ are $(\ERI(\psi), \RI(\psi), c, i)$- and $(\ERI(\chi), \RI(\chi), c, i)$-overlapping.
        By the inductive hypothesis, $(\bar{\mathcal{D}}, \bar{\tau}, v, i) \models \psi$ iff $(\bar{\mathcal{D}}', \bar{\tau}', v, i-c) \models \psi$ and $(\bar{\mathcal{D}}, \bar{\tau}, v, i) \models \chi$ iff $(\bar{\mathcal{D}}', \bar{\tau}', v, i-c) \models \chi$.
        Therefore $(\bar{\mathcal{D}}, \bar{\tau}, v, i) \models \psi \lor \chi$ iff $(\bar{\mathcal{D}}', \bar{\tau}', v, i-c) \models \psi \lor \chi$.
        
    \item 
        $\exists x. \psi$.
        From $\ERI(\exists x. \psi) = \ERI(\psi)$ and $\RI(\exists x. \psi) = \RI(\psi)$ it follows that $(\bar{\mathcal{D}}, \bar{\tau})$ and $(\bar{\mathcal{D}}', \bar{\tau}')$ are $(\ERI(\psi), \RI(\psi), c, i)$-overlapping.
        By the inductive hypothesis we have $(\bar{\mathcal{D}}, \bar{\tau}, v, i) \models \psi$ iff $(\bar{\mathcal{D}}', \bar{\tau}', v, i-c) \models \psi$ for all valuations $v$.
        Thus for all $d$, $(\bar{\mathcal{D}}, \bar{\tau}, v[x \mapsto d], i) \models \psi$ iff $(\bar{\mathcal{D}}', \bar{\tau}', v[x \mapsto d], i-c) \models \psi$.
        By Definition \ref{def:mfotl-evaluation} it follows directly that $(\bar{\mathcal{D}}, \bar{\tau}, v, i) \models \exists x. \psi$ iff $(\bar{\mathcal{D}}', \bar{\tau}', v, i-c) \models \exists x. \psi$ for all valuations $v$.

    \item
        $\Previous{[a,b)} \psi$.
        $(\bar{\mathcal{D}}, \bar{\tau})$ and $(\bar{\mathcal{D}}', \bar{\tau}')$ are $(\ERI(\Previous{[a,b)} \psi), \RI(\Previous{[a,b)} \psi), c, i)$- overlapping, where
        \begin{equation*}
            \ERI(\Previous{[a,b)} \psi) = (-b,0] \Cupext ((-b,-a] \oplusext \ERI(\psi))
        \end{equation*}
        and
        \begin{equation*}
            \RI(\Previous{[a,b)} \psi) = (-b,0] \Cup ((-b,-a] \oplus \RI(\psi))
        \end{equation*}
        From $0 \in \RI(\Previous{[a,b)} \psi)$ and Condition 1 in Definition \ref{def:overlapping-ext}, it follows that $\tau_i = \tau'_{i-c}$.

        We make a case split on the value of $i$. If $i=0$, then $(\bar{\mathcal{D}}, \bar{\tau}, v, i) \not\models \Previous{[a,b)} \psi$, for all valuations $v$.
        From $c \in \mathbb{N}, i-c \in \mathbb{N}$, and $i=0$, it follows that $i-c=0$.
        Since $i-c = 0 = i$ it is trivial that $(\bar{\mathcal{D}}', \bar{\tau}', v, i-c) \not\models \Previous{[a,b)} \psi$ for all valuations $v$.

        Next we consider $i > 0$ and make another case split on whether $\tau_i - \tau_{i-1} \in [a,b)$.
        \begin{itemize}
            \item 
                $\tau_i - \tau_{i-1} \in [a,b)$.
                Then $\tau_{i-1} - \tau_i \in \RI(\Previous{[a,b)} \psi)$ and from Condition 1 in Definition \ref{def:overlapping-ext} it follows that $i-1 \geq c$, $\tau_{i-1} = \tau'_{i-c-1}$, and hence $\tau'_{i-c} - \tau'_{i-c-1} \in [a,b)$.
                It further follows that
                \begin{align*}
                    \ERI(\psi) 
                    &\subseteqmap \{\tau_i - \tau_{i-1} \} \oplusext \{ \tau_{i-1} - \tau_i \} \oplusext \ERI(\psi) \\
                    &\subseteqmap \{\tau_i - \tau_{i-1} \} \oplusext (-b, -a] \oplusext \ERI(\psi) \\
                    &\subseteqmap \{\tau_i - \tau_{i-1} \} \oplusext \ERI(\Previous{[a,b)} \psi)
                \end{align*}
                and
                \begin{align*}
                    \RI(\psi) 
                    &\subseteq \{\tau_i - \tau_{i-1} \} \oplus \{ \tau_{i-1} - \tau_i \} \oplus \RI(\psi) \\
                    &\subseteq \{\tau_i - \tau_{i-1} \} \oplus (-b, -a] \oplus \RI(\psi) \\
                    &\subseteq \{\tau_i - \tau_{i-1} \} \oplus \RI(\Previous{[a,b)} \psi).
                \end{align*}
                Thus by Lemma \ref{lem:eri-overlap-transitivity} $(\bar{\mathcal{D}}, \bar{\tau})$ and $(\bar{\mathcal{D}}', \bar{\tau}')$ are $(\ERI(\psi), \RI(\psi), c, i-1)$- overlapping.
                By the induction hypothesis $(\bar{\mathcal{D}}, \bar{\tau}, v, i-1) \models \psi$ iff $(\bar{\mathcal{D}}', \bar{\tau}', v, i-c-1) \models \psi$ for all valuations $v$.
                Because $\tau_i = \tau'_{i-c}$ and $\tau_{i-1} = \tau'_{i-c-1}$, it follows that $(\bar{\mathcal{D}}, \bar{\tau}, v, i) \models \Previous{[a,b)} \psi$ iff $(\bar{\mathcal{D}}', \bar{\tau}', v, i-c) \models \Previous{[a,b)} \psi$ for all valuations $v$.
            \item 
                $\tau_i - \tau_{i-1} \not\in [a,b)$.
                Then $(\bar{\mathcal{D}}, \bar{\tau}, v, i) \not\models \Previous{[a,b)} \psi$, for all valuations $v$.
                From Condition 1 of Definition \ref{def:overlapping-ext} we still have $i \geq c$.
                We make a case distinction on whether $i=c$ or $i>c$.
                \begin{itemize}
                    \item
                        $i=c$.
                        In this case $i-c = 0$.
                        Therefore $(\bar{\mathcal{D}}, \bar{\tau}, v, i-c) \not\models \Previous{[a,b)} \psi$, for all valuations $v$.
                    \item
                        $i>c$.
                        We proof this case by contradiction.
                        Assume that $\tau'_{i-c} - \tau'_{i-c-1} = \tau'_{i-c} - \tau'_{i-c-1} \in [a,b)$.
                        From Condition 2 in Definition \ref{def:overlapping-ext} it follows that $\tau_{i-1} = \tau'_{i-c-1}$ and hence $\tau_i - \tau_{i-1} = \tau'_{i-c} - \tau'_{i-c-1} \in [a,b)$.
                        This contradicts $\tau_i - \tau_{i-1} \not\in [a,b)$, so it must be the case that $\tau'_{i-c} - \tau'_{i-c-1} \not\in [a,b)$.
                        It follows that $(\bar{\mathcal{D}}', \bar{\tau}', v, i-c) \not\models \Previous{[a,b)} \psi$, for all valuations $v$.
                \end{itemize}
        \end{itemize}
    \item
        $\Next{[a,b)}$.
        The "next" case is in large parts similar to the one for "previous".
        Like with Lemma A.4 \cite{Basin2016} it is simpler than "previous", because no special distinctions have to be made for $i=0$ and $i-c=0$.
    \item
        $\psi \Since{[a,b)} \chi$.
        Like the other cases this one is also very similar to the same case in the proof to Lemma A.4 \cite{Basin2016} and we follow the same structure.
        $(\bar{\mathcal{D}},\bar{\tau})$ and $(\bar{\mathcal{D}}',\bar{\tau}')$ are $(\ERI(\psi \Since{[a,b)} \chi), \RI(\psi \Since{[a,b)} \chi), v, i)$- overlapping, where
        \begin{align*}
            \ERI(\psi \Since{[a,b)} \chi)
            &= (-b,0] \Cupext ((-b,0] \oplusext \ERI(\psi)) \Cupmerge ((-b,-a] \oplusext \ERI(\chi))
        \end{align*}
        and
        \begin{align*}
            \RI(\psi \Since{[a,b)} \chi)
            &= (-b,0] \Cup ((-b,0] \oplus \RI(\psi)) \Cup ((-b,-a] \oplus \RI(\chi)).
        \end{align*}
        From $0 \in \RI(\psi \Since{[a,b)} \chi)$ and Condition 1 in Definition \ref{def:overlapping-ext} it follows that $\tau_i = \tau'_i-c$.

        First we establish the two following claims, by applying the inductive hypothesis to the two sub-formulas $\psi$ and $\chi$.
        \begin{enumerate}
            \item ($\chi$)
                For all $j \in \mathbb{N}$ with $j \leq i$ and $\tau_i - \tau_j \in [a,b)$, it holds that 
                \begin{align*}
                    \RI(\chi) 
                    &\subseteq \{ \tau_i - \tau_j \} \oplus \{ \tau_j - \tau_i\} \oplus \RI(\chi) \\
                    &\subseteq \{\tau_i - \tau_j \} \oplus (-b, -a] \oplus \RI(\chi) \\
                    &\subseteq \{ \tau_i - \tau_j \} \oplus \RI(\psi \Since{[a,b)} \chi)
                \end{align*}
                as well as
                \begin{align*}
                    \ERI(\chi) 
                    &\subseteqmap \{ \tau_i - \tau_j \} \oplusext \{ \tau_j - \tau_i\} \oplusext \ERI(\chi) \\
                    &\subseteqmap \{\tau_i - \tau_j \} \oplusext (-b, -a] \oplusext \ERI(\chi) \\
                    &\subseteqmap \{ \tau_i - \tau_j \} \oplusext \ERI(\psi \Since{[a,b)} \chi)
                \end{align*}
                and $j \geq c$.
                We can convince ourselves that the second step in this equation holds, by considering how the $\oplusext$ operator shifts and or resizes the relative intervals inside a masked predicate map.
                By Lemma \ref{lem:eri-overlap-transitivity} $(\bar{\mathcal{D}}, \bar{\tau})$ and $(\bar{\mathcal{D}}', \bar{\tau}')$ are $(\ERI(\chi), \RI(\chi), c, j)$- overlapping.
                It then follows from the inductive hypothesis that
                \begin{equation*}
                    (\bar{\mathcal{D}}, \bar{\tau}, v, j) \models \chi
                    \text{ iff }
                    (\bar{\mathcal{D}}', \bar{\tau}', v, j-c) \models \chi,
                \end{equation*}
                for all valuations $v$.
            \item ($\psi$)
                For all $k \in \mathbb{N}$ with $k \leq i$ and $\tau_i - \tau_k \in [0,b)$, it holds that
                \begin{align*}
                    \RI(\psi)
                    &\subseteq \{ \tau_i - \tau_k \} \oplus \{ \tau_k - \tau_i \} \oplus \RI(\psi) \\
                    &\subseteq \{ \tau_i - \tau_k \} \oplus (-b,0] \oplus \RI(\psi) \\
                    &\subseteq \{ \tau_i - \tau_k \} \oplus \RI(\psi \Since{[a,b)} \chi) \\
                \end{align*}
                as well as
                \begin{align*}
                    \ERI(\psi)
                    &\subseteqmap \{ \tau_i - \tau_k \} \oplusext \{ \tau_k - \tau_i \} \oplusext \ERI(\psi) \\
                    &\subseteqmap \{ \tau_i - \tau_k \} \oplusext (-b,0] \oplusext \ERI(\psi) \\
                    &\subseteqmap \{ \tau_i - \tau_k \} \oplusext \ERI(\psi \Since{[a,b)} \chi) \\
                \end{align*}
                and $k \geq c$.
                By Lemma \ref{lem:eri-overlap-transitivity} $(\bar{\mathcal{D}}, \bar{\tau})$ and $(\bar{\mathcal{D}}', \bar{\tau}')$ are $(\ERI(\psi), \RI(\psi), c, k)$- overlapping.
                It follows from the inductive hypothesis that
                \begin{equation*}
                    (\bar{\mathcal{D}}, \bar{\tau}, v, k) \models \psi
                    \text{ iff }
                    (\bar{\mathcal{D}}', \bar{\tau}', v, k-c) \models \psi,
                \end{equation*}
                for all valuations $v$.
                
        \end{enumerate}
        We show each direction of the claimed equivalence, $(\bar{\mathcal{D}}, \bar{\tau}, v, i) \models \psi \Since{[a,b)} \chi$ iff $(\bar{\mathcal{D}}', \bar{\tau}', v, i-c) \models \psi \Since{[a,b)} \chi$, for all valuations $v$, separately and start with the direction from left to right.
        If $(\bar{\mathcal{D}}, \bar{\tau}, v, i) \models \psi \Since{[a,b)} \chi$ then there is some $j \leq i$ with $\tau_i - \tau_j \in [a,b)$ such that $(\bar{\mathcal{D}}, \bar{\tau}, v, j) \models \chi$ and $(\bar{\mathcal{D}}, \bar{\tau}, v, k) \models \psi$, for all $k \in [j+1, i+1)$.

        From $\tau_i - \tau_j \in [a,b)$ it follows that $\tau_j - \tau_i \in \RI(\psi \Since{[a,b)} \chi)$ and from Condition 1 in Definition \ref{def:overlapping-ext} it follows that $j \geq c$ and $\tau_j = \tau'_{j-c}$.
        From Claim (i) and from $(\bar{\mathcal{D}}, \bar{\tau}, v, j) \models \chi$ it follows that $(\bar{\mathcal{D}}', \bar{\tau}', v, j-c) \models \chi$.

        For all $k' \in [j+1-c, i+1-c)$, it holds that $\tau'_{k'} - \tau'_{i-c} = \tau'_{k'} - \tau_i \in (-b,0]$ and hence $\tau'_{k'} - \tau_i \in \RI(\psi \Since{[a,b)} \chi)$.
        % TODO continue
        It follows from Condition 2 in Definition \ref{def:overlapping-ext} that $\tau_{k'+c} = \tau'_{k'}$.
        With $(\bar{\mathcal{D}}, \bar{\tau}, v, k' + c) \models \psi$ and Claim (ii) we get $(\bar{\mathcal{D}}', \bar{\tau}', v, k') \models \psi$.
        And with that $(\bar{\mathcal{D}}', \bar{\tau}', v, i-c) \models \psi \Since{[a,b)} \chi$.
        This concludes the proof in the first direction.

        Next we show the other direction of the equivalence, i.e. if $(\bar{\mathcal{D}}', \bar{\tau}', v, i-c) \models \psi \Since{[a,b)} \chi$ then $(\bar{\mathcal{D}}, \bar{\tau}, v, i) \models \psi \Since{[a,b)} \chi$.
        This is equivalent to $(\bar{\mathcal{D}}, \bar{\tau}, v, i) \not\models \psi \Since{[a,b)} \chi$ then $(\bar{\mathcal{D}}', \bar{\tau}', v, i-c) \not\models \psi \Since{[a,b)} \chi$.
        Like in the proof to Lemma A.4 \cite{Basin2016}, we show the second implication.
        From Definition \ref{def:mfotl-evaluation} it follows that there are two possibilities for $(\bar{\mathcal{D}}, \bar{\tau}, v, i) not\models \psi \Since{[a,b)} \chi$.
        The first is that for no $j \leq i$, $\tau_i - \tau_j \in [a,b)$ and $(\bar{\mathcal{D}}, \bar{\tau}, v, i) \models \chi$.
        Or written differently, for all $j \leq i$ with $\tau_i - \tau_j \in [a,b)$, $(\bar{\mathcal{D}}, \bar{\tau}, v, i) \not\models \chi$.
        From the Definition \ref{def:rel-int} it follows that for all $j' \leq i-c$ with $\tau'_{i-c} - \tau'_{j'} = \tau_i - \tau'_{j'} \in [a,b)$, $\tau'_{j'} - \tau_i \in \RI(\psi \Since{[a,b)} \chi)$.
        This is a useful property, because it means we can conclude from Condition 2 of Definition \ref{def:overlapping-ext} that $\tau_{j'+c} = \tau'_{j'}$ for all $j' \leq i-c$.
        And as in the proof to Lemma A.4 \cite{Basin2016} this means that there cannot be a time stamp within the interval $[a,b)$ in $(\bar{\mathcal{D}}', \bar{\tau}')$ that is not present in $(\bar{\mathcal{D}}, \bar{\tau})$.
        We are currently looking at the case where for all $j \leq i$, $(\bar{\mathcal{D}}, \bar{\tau}, v, j) \not\models \chi$.
        Thus also $(\bar{\mathcal{D}}, \bar{\tau}, v, j'+c) \not\models \chi$ for all $j' \leq i-c$.
        Combining this with Claim (i) from above we get that $(\bar{\mathcal{D}}', \bar{\tau}', v, j') \not\models \chi$.
        And by Definition \ref{def:mfotl-evaluation} it follows that $(\bar{\mathcal{D}}', \bar{\tau}', v, i-c) \not\models \psi \Since{[a,b)} \chi$.

        The second possibility for $(\bar{\mathcal{D}}, \bar{\tau}, v, i) not\models \psi \Since{[a,b)} \chi$ is, that for all $j \leq i$ with $\tau_i - \tau_j \in [a,b)$ and $(\bar{\mathcal{D}}, \bar{\tau}, v, j) \models \chi$ there is some $k \in \mathbb{N}$ with $j < k \leq i$, $(\bar{\mathcal{D}}, \bar{\tau}, v, k) \not\models \psi$.
        It follows from this that for every $j' \in \mathbb{N}$ with $j' \leq i-c$, $\tau'_{i-c} - \tau'_{j'} \in [a,b)$, and $(\bar{\mathcal{D}}', \bar{\tau}', v, j') \models \chi$, there exists a $j \in \mathbb{N}$ with $j = j' + c$.





        




        \dots


    \item
        $\psi \Until{[a,b)} \chi$.
        The "Until" case is analogous to the "Since" case.
\end{itemize}