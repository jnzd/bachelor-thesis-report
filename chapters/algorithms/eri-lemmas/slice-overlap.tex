\begin{lemma}
    \label{lem:mt-slice-overlap}
    Let $T \subseteq \mathbb{N}$ and $I \subseteq \mathbb{Z}$ be intervals, 
    $(\bar{\mathcal{D}}, \bar{\tau})$ a temporal structure, and $(\bar{\mathcal{D}}', \bar{\tau}')$ a $(T \oplusext M, T \oplus I)_{\tau_i}$- slice of $(\bar{\mathcal{D}}, \bar{\tau})$.
    The temporal structures $(\bar{\mathcal{D}}, \bar{\tau})$ and $(\bar{\mathcal{D}}', \bar{\tau}')$ are $(M,I,c,i)$- overlapping, for all $i \in \mathbb{N}$ with $\tau_i \in T$, where $c \in \mathbb{N}$ is the value in Definition \ref{def:overlapping-ext} used by the function $s$ with respect to $(\bar{\mathcal{D}}, \bar{\tau})$ and its slice $(\bar{\mathcal{D}}', \bar{\tau}')$.
\end{lemma}
\textit{Proof}
This proof follows the same structure as the proof to Lemma A.2 in Basin et al. \cite{Basin2016} and the first few steps are the same.
The precondition in Definition \ref{def:overlapping-ext}, that every interval in $M$ is contained in the interval $T$, is satisfied, as it is also a precondition for Lemma \ref{lem:mt-slice-overlap}.
First we show condition 1 in Definition \ref{def:overlapping-ext}.
For all $j \in \mathbb{N}$ with $\tau_j - \tau_i \in T$, it follows from $\tau_i \in I$ that $\tau_j \in T \oplus I$.
From $c = \min\{k \in \mathbb{N} \mid \tau_k \in T \}$ we get $j \geq c$.
Even more precisely  for all $j \in \mathbb{N}$ with $\tau_j \in T \oplus I$, we get that $j \in [c, l+c)$.
From Definition \ref{def:mt-slice} it follows that
\begin{align*}
    \mathcal{D}'_{j-c}
    & = \filter(T \oplusext M, \mathcal{D}_{s(j-c)}, \tau_{s(j-c)}, \tau_i) \\
    & = \filter(T \oplusext M, \mathcal{D}_{j-c+c}, \tau_{j-c+c}, \tau_i) \\
    & = \filter(T \oplusext M, \mathcal{D}_{j}, \tau_{j}, \tau_i), 
\end{align*}
for all $j \in \mathbb{N}$ with $\tau_j - \tau_i \in T \oplus I$.
With the help of Lemma \ref{lem:oplusext-zero} and Definition \ref{def:filter} (shifting all intervals in a map, shifts all time stamps that fall into these intervals the same way) we can convince ourselves that the above fact implies
\begin{align*}
    \mathcal{D}'_{j-c}
    & = \filter((-T \oplus T) \oplusext M, \mathcal{D}_{s(j-c)}, \tau_{s(j-c)}, \tau_i) \\
    & = \filter(\{0\} \oplusext M, \mathcal{D}_{j-c+c}, \tau_{j-c+c}, \tau_i) \\
    & = \filter(M, \mathcal{D}_{j}, \tau_{j}, \tau_i), 
\end{align*}
for all $j \in \mathbb{N}$ with $\tau_j - \tau_i \in (-T \oplus T) \oplus I = I$.
Thus we get
\begin{align*}
    \mathcal{D}'_{j-c}
    & = \filter(M, \mathcal{D}_{j}, \tau_{j}, \tau_i), 
\end{align*}
for all $i \in \mathbb{N}$ with $\tau_i \in T$, and $j \in \mathbb{N}$ with $\tau_j - \tau_i \in I$.
From Lemma \ref{lem:double-filter} it follows that
\begin{align*}
    \mathcal{D}'_{j-c}
    &= \filter(M, \mathcal{D}_j, \tau_j, \tau_i) \\
    &= \filter(M, \filter(M, \mathcal{D}_j, \tau_j, \tau_i), \tau_j, \tau_i) \\
    &= \filter(M, \mathcal{D}'_{j-c}, \tau_j, \tau_i) \\
    &= \filter(M, \mathcal{D}'_{j-c}, \tau'_{j-c}, \tau_i) \\
\end{align*}
for all $j \in \mathbb{N}$.
And with that we have shown everything for Condition 1 in Definition \ref{def:overlapping-ext}.

For Condition 2 we can deduce that for all $i \in \mathbb{N}$ with $\tau_i \in T$ and for all $j' \in \mathbb{N}$ with $\tau'_{j'} - \tau_i \in I$, $\tau'_{j'} \in T \oplus I$.
And because $\tau'_l \not\in T \oplus I$, $j' \in [0,l)$.
Hence $\tau_{j'+c} = \tau_{s(j')} = \tau'_{j'}$.
From Definition \ref{def:mt-slice} it follows that
\begin{align*}
    \mathcal{D}'_{j'}
    & = \filter(T \oplusext M, \mathcal{D}_{s(j')}, \tau_{s(j')}, \tau_i) \\
    & = \filter(T \oplusext M, \mathcal{D}_{j'+c}, \tau_{j'+c}, \tau_i) \\
\end{align*}
for all $j' \in \mathbb{N}$ with $\tau_j' - \tau_i \in T \oplus I$.
Again with the help of Lemma \ref{lem:oplusext-zero} and Definition \ref{def:filter} we can convince ourselves that the above fact implies
\begin{align*}
    \mathcal{D}'_{j'}
    & = \filter((-T \oplus T) \oplusext M, \mathcal{D}_{s(j')}, \tau_{s(j')}, \tau_i) \\
    & = \filter(\{0\} \oplusext M, \mathcal{D}_{j'+c}, \tau_{j'+c}, \tau_i) \\
    & = \filter(M, \mathcal{D}_{j'+c}, \tau_{j'+c}, \tau_i), 
\end{align*}
for all $j' \in \mathbb{N}$ with $\tau_j' - \tau_i \in (-T \oplus T) \oplus I = I$.
Thus we get
\begin{align*}
    \mathcal{D}'_{j'}
    & = \filter(M, \mathcal{D}_{j'+c}, \tau_{j+c}, \tau_i), 
\end{align*}
And like with condition 1, from Lemma \ref{lem:double-filter} it follows that
\begin{align*}
    \mathcal{D}'_{j'}
    & = \filter(M, \mathcal{D}_{j'+c}, \tau_{j'+c}, \tau_i)\\
    & = \filter(M, \mathcal{D}_{j'+c}, \tau_{j'+c}, \tau_i)\\ 
    & = \filter(M, \filter(M, \mathcal{D}_{j'+c}, \tau_{j'+c}, \tau_i), \tau_{j'+c}, \tau_i) \\
    & = \filter(M, \mathcal{D}'_{j'}, \tau_{j'+c}, \tau_i) \\
    & = \filter(M, \mathcal{D}'_{j'}, \tau'_{j'}, \tau_i),\\
\end{align*}
for all $j' \in \mathbb{N}$.
And with that we have shown everything for Condition 2 in Definition \ref{def:overlapping-ext}.