\begin{center}
\textbf{Abstract}
\end{center}

Runtime Monitoring is the process of observing a system's behavior and checking it against a formal specification.
One tool to perform runtime monitoring is MonPoly, which uses metric first order temporal logic (MFOTL) as a policy specification language.
MonPoly evaluates MFOTL formulas over a timestamped logs of events, or \emph{traces}, that capture the system's behavior.

Natively, the MonPoly tool supports online monitoring only via standard input and on a single machine,
and does not provide a user-friendly interface to access the events it keeps track of in its state.
Moreover, MonPoly provides no mechanism for updating or changing the policy being monitored.
In this talk, we present a new REST-based wrapper that combines MonPoly with the time series database QuestDB to provide a flexible interface supporting both (persistent) logging and (online) monitoring
of events over the network.

Our wrapper also supports policy change by starting a new instance of MonPoly and initializing its state using a fragment of the persistent logs.
We present an algorithm that significantly reduces the size of the log fragment that needs to be extracted to initialize the new state.
Our algorithm builds on an extension of the notion of relative intervals of MFOTL formulas introduced by Basin, Caronni, et al. 